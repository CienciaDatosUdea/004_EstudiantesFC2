% Created 2025-08-25 Mon 17:46
% Intended LaTeX compiler: lualatex
\documentclass[11pt]{article}
\usepackage{amsmath}
\usepackage{fontspec}
\usepackage{graphicx}
\usepackage{longtable}
\usepackage{wrapfig}
\usepackage{rotating}
\usepackage[normalem]{ulem}
\usepackage{capt-of}
\usepackage{hyperref}
\usepackage[outputdir=./build]{minted}
\date{\today}
\title{}
\hypersetup{
 pdfauthor={},
 pdftitle={},
 pdfkeywords={},
 pdfsubject={},
 pdfcreator={Emacs 30.2 (Org mode 9.8-pre)},
 pdflang={English}}
\begin{document}

\tableofcontents

J.D. Salcedo.

\noindent\rule{\textwidth}{0.5pt}

Proyecto Final Física Computacional 2.
¿Dónde están las partículas cuando la caja está caliente? </b> </h1>

\noindent\rule{\textwidth}{0.5pt}
\section{\textbf{Resumen}}
\label{sec:org88c7a85}
Este proyecto aborda el problema de localizar partículas, como fermiones
y bosones (con y sin espín), dentro de una barrera de potencial finita
de ancho \(a\), contenida a su vez dentro de una barrera de potencial
infinita de ancho \(b\), donde \(b >> a\). El sistema es influenciado
por un reservorio térmico que lo calienta. Además, se estudió el caso
límite cuando \(a = b\), obteniendo resultados que coinciden con los
reportados en la literatura.

\noindent\rule{\textwidth}{0.5pt}
\section{\textbf{1. Introducción}}
\label{sec:orgdaa2dc7}
Estudiamos una partícula cuántica sujeta a un potencial que generaliza
al bien conocido caso del pozo infinito. El sistema cuántico propuesto
constituye una manera de cuantizar la región de energías continuas del
también caracterizado caso del pozo finito; esto resulta de especial
importancia en los estudios que se desean llevar a cabo en el presente
trabajo, pues se trata de emplear el formalismo de la colectividad
canónica de Gibbs para dotar al espacio de estados energéticos del pozo
de una densidad de probabilidad estadísitica que haga referencia al
contacto con un reservorio de calor; dado que se ha cuantizado el
continuo, es entonces posible asignar de manera unívoca una función de
onda a cada nivel energético del sistema. Por tanto, es lícito
preguntarse por el valor esperado del ahora bien definido observable
que, en cada estado \(n\) de energía, evalúa el valor de la densidad de
probabilidad posicional (cuántica) \(|\phi_n(x)|^2\), para un \(x\) fijo
dentro del pozo. La colección de estos valores esperados para todo \(x\)
da lugar a una función que interpretamos como la densidad de
probabilidad termalizada del sistema.

En este trabajo, resolvemos e indagamos los aspectos cuánticos y las
propiedades térmicas del sistema para una sola partícula y para bosones
y fermiones; hallamos de forma sistemática los niveles energéticos y sus
correspondientes funciones de onda teniendo en cuenta el formalismo
cuántico [2,3] y hallamos los valores esperados a la luz
de la teoría estadística del calor [4]. Como resultado particular
de nuestro tratamiento, se obtienen los mismos resultados que ya han
sido reportados en la literatura para un pozo infinito usual [1].

\noindent\rule{\textwidth}{0.5pt}

\begin{minted}[breaklines=true,breakanywhere=true,frame=lines]{python}
!apt-get install libgsl-dev
!apt-get install libeigen3-dev
\end{minted}

\phantomsection
\label{org05e4215}
\begin{verbatim}
zsh:1: command not found: apt-get
zsh:1: command not found: apt-get
\end{verbatim}

\begin{minted}[breaklines=true,breakanywhere=true,frame=lines]{python}
!pip install tabulate
\end{minted}

\phantomsection
\label{orge05e2ea}
\begin{verbatim}
zsh:1: command not found: pip
\end{verbatim}

\begin{minted}[breaklines=true,breakanywhere=true,frame=lines]{python}
!pip install pandas
\end{minted}

\phantomsection
\label{orgce4cc65}
\begin{verbatim}
zsh:1: command not found: pip
\end{verbatim}

\begin{minted}[breaklines=true,breakanywhere=true,frame=lines]{python}
!pip install sympy numpy scipy
\end{minted}

\phantomsection
\label{org647205d}
\begin{verbatim}
zsh:1: command not found: pip
\end{verbatim}

\begin{minted}[breaklines=true,breakanywhere=true,frame=lines]{python}
#Usual packages
%config InlineBackend.figure_format = 'svg'

from IPython.core.interactiveshell import InteractiveShell
from itertools import combinations_with_replacement
from scipy.interpolate import interp1d
from scipy.optimize import ridder
from scipy.integrate import quad
import matplotlib.pyplot as plt
import pandas as pd
import sympy as sym
import numpy as np
import subprocess
import itertools
from tabulate import tabulate

InteractiveShell.ast_node_interactivity = "all"
plt.rcParams.update({
    'mathtext.fontset': 'stix',
    'font.family': 'STIXGeneral'
})

# A function to easily display datasets in orgmode
def print_table(dataframe):
    # Display a given dataset in a table format
    table = tabulate(dataframe, headers='keys', tablefmt='orgtbl')
    print(table)
\end{minted}

\noindent\rule{\textwidth}{0.5pt}
\section{\textbf{2. Marco Teórico}}
\label{sec:orgd7fc039}
\subsection{\textbf{2.1 Cuantización de la energia}}
\label{sec:orgda3cdd4}
Sea el potencial dado por
$$
 V(x) = \begin{cases} 0, \quad 0 < x < a,\\
                      V_0, \quad a < x < b,\\
                      \infty, \quad x > b,
 \end{cases}
 $$
donde \(V_0\) es una constante y el potencial es par, es decir, \(V(-x)
= V(x)\).

Siguiendo la metodología de [3], dividimos el problema en tres regiones, la
función de onda en cada una de ellas, para \(j = 0,1,2\), se escribe como
$$ \psi_j = A_j e^{ik_j x} + B_j e^{-ik_j x},
$$
donde
$$ k_j = \sqrt{\frac{2m(E - V_j)}{\hbar^2}} $$
y \(A_j\), \(B_j\) son constantes de integración.

Para relacionar las ondas entre las regiones \(0\) y \(1\), imponemos la
continuidad de la función de onda y su derivada \((C^1)\) en \(x=a\).
Esto nos permite definir la matriz de transferencia como
$$
 M_{01}(-a) = \frac{1}{2k_0} \begin{pmatrix} s_{01} e^{id_{01}a} & d_{01} e^{is_{01}a} \\
                                             d_{01} e^{-is_{01}a} & s_{01}e^{-id_{01}a}
                             \end{pmatrix},
$$
junto con el vector de coeficientes
$$
C_j = \begin{pmatrix} A_j \\ B_j \end{pmatrix}, \quad j = 0,1,2.
 $$
Aquí,
$$ s_{01} = k_0 + k_1, \quad d_{01} = k_0 - k_1.
$$
Así, las ondas a cada lado del salto de potencial están relacionadas por
$$ C_0 = M_{01}(-a)C_1. $$

De manera análoga, para la interfaz entre las regiones \(1\) y \(2\), se
obtiene la relación
$$ C_1 = M_{12}(a)C_2, $$
donde
$$
 M_{12}(a) = \frac{1}{2k_0} \begin{pmatrix} s_{12} e^{-id_{12}a} & d_{12} e^{-is_{12}a} \\
                                            d_{12} e^{is_{12}a} & s_{12}e^{id_{12}a}
                            \end{pmatrix}.
$$

Combinando estas relaciones, se obtiene
$$ C_0 = M_{01}(-a)M_{12}(a)C_2. $$

Dado que la función de onda debe anularse en los extremos, \(\psi_0(-b) = 0\) y
\(\psi_2(b) = 0\), se deduce que
$$ A_0 = - B_0 e^{2ik_0b}, \quad A_2 = - B_2 e^{-2ik_0b}.
$$

Sustituyendo en la ecuación anterior, se obtiene
$$
 \begin{pmatrix} B_0 e^{2ik_0b} \\ B_0 \end{pmatrix} =
 M_{01}(-a) M_{12}(a) \begin{pmatrix} B_2 e^{-2ik_0b} \\ B_2 \end{pmatrix}.
$$

De esta ecuación se obtienen dos relaciones, las cuales se dividen entre
sí para obtener la ecuación trascendental. Debido a la extensión de los
cálculos, esta ecuación será implementada en Python utilizando
manipulación simbólica para realizar las operaciones correspondientes.
\begin{minted}[breaklines=true,breakanywhere=true,frame=lines]{python}
# Finding the transcendental equation...

# Define symbolic variables
k_0, k_1 = sym.symbols('k_0 k_1', real=True)
a, b = sym.symbols('a b', real=True)
i = sym.I  # Imaginary unit

# Define expressions
s01 = k_0 + k_1
d01 = k_0 - k_1
s12 = k_1 + k_0
d12 = k_1 - k_0

# Define matrices
M_01 = sym.Matrix([[s01 * sym.exp(i * d01 * a), d01 * sym.exp(i * s01 * a)],
                   [d01 * sym.exp(-i * s01 * a), s01 * sym.exp(-i * d01 * a)]])

M_12 = sym.Matrix([[s12 * sym.exp(-i * d12 * a), d12 * sym.exp(-i * s12 * a)],
                   [d12 * sym.exp(i * s12 * a), s12 * sym.exp(i * d12 * a)]])

# Multiply the matrices
M = (M_01 @ M_12).expand()

# Vector multiplication
vect = M @ sym.Matrix([-sym.exp(-2 * i * k_0 * b), 1])

#...
x = sym.trigsimp(vect[0].expand().rewrite(sym.sin), method='fu').simplify()
x = sym.collect(x, [k_1 ** 2, k_0 ** 2,k_0*k_1,2*k_1, 2*k_0])

#...
y = sym.trigsimp(vect[1].rewrite(sym.cos), method='fu')
y = y.simplify()
y = sym.collect(y, [k_1 ** 2, k_0 ** 2,k_0*k_1])


#...
function = -sym.exp(2 * i * k_0 * b) * y - x
function = function.rewrite(sym.sin)

#...
function1 = sym.trigsimp(sym.expand_trig(function).expand(), method='fu')
function1 = sym.collect(function1, [2*i*k_1 ** 2, 2*i*k_0 ** 2,2*i*k_0*k_1])
function1
\end{minted}

\phantomsection
\label{org79beec0}
\(\displaystyle 2 i k_{0}^{2} \left(- 2 \sin{\left(2 a k_{1} \right)} - \sin{\left(- 2 a k_{0} + 2 a k_{1} + 2 b k_{0} \right)} - \sin{\left(2 a k_{0} + 2 a k_{1} - 2 b k_{0} \right)}\right) + 2 i k_{0} k_{1} \left(- 2 \sin{\left(- 2 a k_{0} + 2 a k_{1} + 2 b k_{0} \right)} + 2 \sin{\left(2 a k_{0} + 2 a k_{1} - 2 b k_{0} \right)}\right) + 2 i k_{1}^{2} \left(2 \sin{\left(2 a k_{1} \right)} - \sin{\left(- 2 a k_{0} + 2 a k_{1} + 2 b k_{0} \right)} - \sin{\left(2 a k_{0} + 2 a k_{1} - 2 b k_{0} \right)}\right)\)

A continuación, simplificamos la expresión previa dividiéndola en tres
partes:

Para el primer término, se obtiene:
$$ \begin{split} -k_0^2 \left(2 \sin(2 a k_1) + \sin\left(
2ak_1 - 2 k_0 (a - b) \right) + \sin\left( 2ak_1 + 2 k_0 (a - b) \right)
\right) & = -k_0^2 \left(2 \sin(2 a k_1) + 2 \sin(2 a k_1) \cos(2 k_0
(a - b)) \right) \ & = -2 k_0^2 \sin(2 a k_1) \left(1 + \cos(2 k_0 (a -
b)) \right). \end{split} $$

Para el segundo término, tenemos:
$$ \begin{split} 2k_0 k_1 \left(-\sin\left(2 a k_1 - 2k_0
(a-b) \right) + \sin\left( 2ak_1 + 2 k_0 (a - b) \right) \right) & =
2k_0 k_1 \left(2 \cos(2 a k_1) \sin(2 k_0 (a - b)) \right) \ & = 4 k_0
k_1 \cos(2 a k_1) \sin(2 k_0 (a - b)). \end{split} $$

Finalmente, para el tercer término:
$$ \begin{split} -k_1^2 \left(-2 \sin(2 a k_1) +
\sin\left( 2ak_1 - 2 k_1 (a - b) \right) + \sin\left( 2ak_1 + 2 k_1 (a -
b) \right) \right) & = -k_1^2 \left(-2 \sin(2 a k_1) + 2 \sin(2 a k_1)
\cos(2 k_1 (a - b)) \right) \ & = 2 k_1^2 \sin(2 a k_1) \left(1 - \cos(2
k_1 (a - b)) \right). \end{split} $$

Sumando los tres términos obtenemos la ecuación trascendental:
$$ \boxed{0 = (k_1^2 - k_0^2)\sin(2 a k_1) + 2 k_0 k_1
\cos(2 a k_1) \sin(2 k_0 (a - b)) -(k_1^2 + k_0^2) \sin(2 a k_1) \cos(2
k_0 (a - b)) } $$

Este resultado nos permite calcular los valores de energía superiores al
potencial \(V_0\), es decir, aquellos para los cuales \(E > V_0\). Para
obtener la ecuación trascendental correspondiente a los valores de
energía inferiores a \(V_0\), realizamos el cambio de variable \(k_0 =
ip\),
donde \(p = \sqrt{2m(V_0 - E)/\hbar}\), y utilizamos las identidades hiperbólicas:
$$
\cos(ix) = \cosh(x), \quad \sin(ix) = i \sinh(x).
$$

Sustituyendo estas expresiones en la ecuación trascendental, obtenemos:
$$ \boxed{0 = (k_1^2 + p^2)\sin(2ak_1) - 2pk_1 \cos(2ak_1)
\sinh(2p(a - b)) -(k_1^2 - p^2) \sin(2 a k_1) \cosh(2p (a - b))}
$$

Ahora que hemos obtenido las ecuaciones trascendentales que cuantizan la
energía en todas las regiones del pozo, procederemos a traducirlas a un
lenguaje simbólico en Python para encontrar sus soluciones numéricas.

\begin{minted}[breaklines=true,breakanywhere=true,frame=lines]{python}
# Define parameters to illustrate implemented methodology

m_val     = 1.0  # Particle mass
V_0_val   = 5.0  #
h_bar_val = 1    #
a_val     = 1.5  # Small well width
b_val     = 3.0  # Large well width
T         = 1    # Temperature

# Case E > V_0
term1 = (k_1 **2 - k_0 ** 2) * sym.sin(2 * k_1 * a)
term2 = 2 * k_0 * k_1 * sym.sin(2 * k_0 * (a - b)) * sym.cos(2 * k_1 * a)
term3 = -(k_0 ** 2 + k_1 ** 2) * sym.cos(2 * k_0 * (a - b)) * sym.sin(2 * k_1 * a)

function = term1 + term2 + term3

E, m, h_bar, V_0 = sym.symbols('E m h_bar V_0')

# Let k_i = sqrt(2 * m * (E - V_i)) / h_bar
function = function.subs(k_0, sym.sqrt(2 * m * (E - V_0)) / h_bar)
function = function.subs(k_1, sym.sqrt(2 * m * E) / h_bar)

# Convert to a numerical function
numeric_function = sym.lambdify((E, m, V_0, h_bar, a, b), function, 'numpy')

# Case E < V_0
q = sym.symbols('q') # k_0 = i * q; q a real number
term1 = (k_1 **2 + q ** 2) * sym.sin(2 * k_1 * a)
term2 = -2 * q * k_1 * sym.sinh(2 * q * (a - b)) * sym.cos(2 * k_1 * a)
term3 = (q ** 2 - k_1 ** 2) * sym.cosh(2 * q * (a - b)) * sym.sin(2 * k_1 * a)

function2 = term1 + term2 + term3

E, m, h_bar, V_0 = sym.symbols('E m h_bar V_0')

# Let k = sqrt(2 * m * E) / h_bar
function2 = function2.subs(q, sym.sqrt(2 * m * (V_0 - E)) / h_bar)
function2 = function2.subs(k_1, sym.sqrt(2 * m * E) / h_bar)

# Convert to numerical function
numeric_function2 = sym.lambdify((E, m, V_0, h_bar, a, b), function2, 'numpy')

if V_0_val == 0:

  # Energy range
  E_values = np.linspace(V_0_val + 0.01, V_0_val+ 1000, 100000)  # Avoid division by zero at E = V_0

  # Compute function values
  function_values = numeric_function(E_values, m_val, V_0_val, h_bar_val, a_val, b_val)

  plt.figure(figsize=(10, 5))
  plt.plot(E_values, function_values, label=r'$f(E > V_0)$', color='black', linestyle='-')
  plt.xlabel(r'Energy $E$')
  plt.ylabel(r'$f(E)$')
  plt.title(r'Transcendental Function $f(E > V_0)$')
  plt.grid()
  plt.legend()
  plt.show();

else:
    # Energy range
  E_values = np.linspace(V_0_val + 0.01, V_0_val+ 1000, 100000)  # Avoid division by zero at E = V_0

  # Energy range
  E_values2 = np.linspace(0, V_0_val - 1e-3,100000)  # Avoid division by zero at E = V_0

  # Compute function values
  function_values = numeric_function(E_values, m_val, V_0_val, h_bar_val, a_val, b_val)

  # Compute function values
  function_values2 = numeric_function2(E_values2, m_val, V_0_val, h_bar_val, a_val, b_val)
  fig, axs = plt.subplots(1, 2, figsize=(12, 5))
  ax = axs.flatten()

  ax[0].plot(E_values2, function_values2, label=r'$f(E < V_0)$', color='black', linestyle='-')
  ax[0].set_xlabel(r'Energy $E$')
  ax[0].set_ylabel(r'$f(E)$')
  ax[0].set_title(r'Transcendental Function $f(E < V_0)$')
  ax[0].grid()
  ax[0].legend()

  ax[1].plot(E_values, function_values, label=r'$f(E > V_0)$', color='black', linestyle='-')
  ax[1].set_xlabel(r'Energy $E$')
  ax[1].set_ylabel(r'$f(E)$')
  ax[1].set_title(r'Transcendental Function $f(E > V_0)$')
  ax[1].legend()
  ax[1].grid()

  plt.tight_layout()
  plt.show();
\end{minted}

\phantomsection
\label{org9799f51}
\begin{center}
\begin{tabular}{lll}
<matplotlib.lines.Line2D & at & 0x7f73906ab610>\\
\end{tabular}
\end{center}
\begin{verbatim}
Text(0.5, 0, 'Energy $E$')
Text(0, 0.5, '$f(E)$')
Text(0.5, 1.0, 'Transcendental Function $f(E < V_0)$')
<matplotlib.legend.Legend at 0x7f73907cf770>
\end{verbatim}

\begin{center}
\begin{tabular}{lll}
<matplotlib.lines.Line2D & at & 0x7f739074c190>\\
\end{tabular}
\end{center}
\begin{verbatim}
Text(0.5, 0, 'Energy $E$')
Text(0, 0.5, '$f(E)$')
Text(0.5, 1.0, 'Transcendental Function $f(E > V_0)$')
<matplotlib.legend.Legend at 0x7f739074c2d0>
\end{verbatim}

\begin{center}
\includesvg[width=.9\linewidth]{./.ob-jupyter/f19cdbdbebc1febd8115997b3f390137414cde67}
\end{center}

Ahora, aplicaremos un método numérico para encontrar los ceros de la
función representada en la gráfica anterior. Estos ceros corresponden a
los niveles de cuantización de la energía en toda la región considerada.

\begin{minted}[breaklines=true,breakanywhere=true,frame=lines]{python}
# Computing the roots for this transcendental equation
def find_zeros_ridder_array(x, f, tol=1e-6):
    """
    Finds multiple zeros of a function defined by discrete (x, f) data using Ridder's method.

    Parameters:
    - x: array-like, x-values of the function.
    - f: array-like, corresponding function values f(x).
    - tol: float, optional, tolerance for stopping criteria.

    Returns:
    - list of floats, estimated roots.
    """
    try:
        # Interpolating function
        f_interp = interp1d(x, f, kind='cubic', fill_value="extrapolate")

        roots = []
        # Scan for sign changes (potential root intervals)
        for i in range(len(x) - 1):
            if f[i] * f[i + 1] < 0:  # Sign change detected
                root = ridder(f_interp, x[i], x[i + 1], xtol=tol)
                roots.append(root)

        if not roots:
            print("No roots found in the given range.")

        return roots
    except Exception as e:
        print(f"Error finding zeros: {e}")
        return []


def Energy_values(m_val, V_0_val, h_bar_val, a_val, b_val):
    """
    Computes the energy values for a quantum well system using numerical methods.

    Parameters:
    m_val      : float - Particle mass
    V_0_val    : float - Potential well depth
    h_bar_val  : float - Reduced Planck’s constant
    a_val      : float - Width of the inner well
    b_val      : float - Width of the outer well

    Returns:
    root(s)    : np.array - Sorted array of energy values that satisfy the quantization condition.
    """

    if V_0_val == 0:
        # Define energy range for an infinite potential well case
        E_values = np.linspace(V_0_val + 0.01, V_0_val + 1000, 100000)  # Avoid division by zero at E = V_0

        # Compute function values for energy equation
        function_values = numeric_function(E_values, m_val, V_0_val, h_bar_val, a_val, b_val)

        # Find energy values that satisfy the equation (roots of the function)
        root = find_zeros_ridder_array(E_values, function_values)
        return root

    else:
        # Define energy range for bound states (E > V_0)
        E_values = np.linspace(V_0_val + 0.01, V_0_val + 1000, 100000)  # Avoid division by zero at E = V_0

        # Define energy range for bound states (E < V_0)
        E_values2 = np.linspace(0, V_0_val - 1e-3, 100000)  # Avoid division by zero at E = V_0

        # Compute function values for E > V_0
        function_values = numeric_function(E_values, m_val, V_0_val, h_bar_val, a_val, b_val)

        # Compute function values for E < V_0
        function_values2 = numeric_function2(E_values2, m_val, V_0_val, h_bar_val, a_val, b_val)

        # Find energy values that satisfy the equation (roots of the function)
        root = find_zeros_ridder_array(E_values, function_values)
        root2 = find_zeros_ridder_array(E_values2, function_values2)

        # Combine and return all valid energy values in sorted order
        return np.sort(np.hstack([root, root2]))


if V_0_val == 0:
    root = find_zeros_ridder_array(E_values, function_values)
    #root2 = find_zeros_ridder_array(E_values, function_values_varied)
    plt.figure(figsize=(12, 5))
    plt.plot(E_values, function_values, label=r'$f(E)$', color='black', linestyle='-')
    plt.scatter(root, np.zeros_like(root), color='red', marker='o', label='Root')
    #plt.plot(E_values, function_values_varied, label=r'$f(E)$', color='red', linestyle='--')
    plt.xlabel(r'Energy $E$')
    plt.ylabel(r'$f(E)$')
    plt.title(r'Function vs. Energy $(E > V_0)$')
    plt.grid()
    plt.legend()
    plt.show()

else:
  root = find_zeros_ridder_array(E_values, function_values)
  root2 = find_zeros_ridder_array(E_values2, function_values2)
  fig, axs = plt.subplots(1, 2, figsize=(12, 5))
  ax = axs.flatten()

  ax[0].plot(E_values2, function_values2, label=r'$f(E)$', color='black', linestyle='-')
  ax[0].scatter(root2, np.zeros_like(root2), color='red', marker='o', label='Root')
  ax[0].set_xlabel(r'Energy $E$')
  ax[0].set_ylabel(r'$f(E)$')
  ax[0].set_title(r'Function vs. Energy $(E < V_0)$')
  #ax[0].axhline(0, color='gray', linestyle='dotted', linewidth=0.8)  # Zero reference line
  ax[0].grid()
  ax[0].legend()

  ax[1].plot(E_values, function_values, label=r'$f(E)$', color='black', linestyle='-')
  ax[1].scatter(root, np.zeros_like(root), color='red', marker='o', label='Root')
  ax[1].set_xlabel(r'Energy $E$')
  ax[1].set_ylabel(r'$f(E)$')
  ax[1].set_title(r'Function vs. Energy $(E > V_0)$')
  #ax[1].set_xlim(50, 60)
  #ax[1].ylim(-0.5e4, 0.5e4)
  #ax[1].axhline(0, color='gray', linestyle='dotted', linewidth=0.8)  # Zero reference line
  ax[1].legend()
  ax[1].grid()

  plt.tight_layout()
  plt.show()
\end{minted}

\phantomsection
\label{orge78ec3e}
\begin{center}
\begin{tabular}{lll}
<matplotlib.lines.Line2D & at & 0x7f739092b610>\\
\end{tabular}
\end{center}
\begin{verbatim}
<matplotlib.collections.PathCollection at 0x7f73906fda90>
Text(0.5, 0, 'Energy $E$')
Text(0, 0.5, '$f(E)$')
Text(0.5, 1.0, 'Function vs. Energy $(E < V_0)$')
<matplotlib.legend.Legend at 0x7f739092b750>
\end{verbatim}

\begin{center}
\begin{tabular}{lll}
<matplotlib.lines.Line2D & at & 0x7f73908ecf50>\\
\end{tabular}
\end{center}
\begin{verbatim}
<matplotlib.collections.PathCollection at 0x7f73908ed090>
Text(0.5, 0, 'Energy $E$')
Text(0, 0.5, '$f(E)$')
Text(0.5, 1.0, 'Function vs. Energy $(E > V_0)$')
<matplotlib.legend.Legend at 0x7f73908ed450>
\end{verbatim}

\begin{center}
\includesvg[width=.9\linewidth]{./.ob-jupyter/62e6a8e70ba6942dd58410d5bcc63304ed48f33e}
\end{center}

Como se observa en la gráfica anterior, hemos localizado los niveles de
energía cuantizados. Esto nos permitirá avanzar con los cálculos
expuestos más adelante. Por ahora, podemos comparar nuestros resultados
con los obtenidos teóricamente para un pozo infinito de ancho \(2a\).

\begin{minted}[breaklines=true,breakanywhere=true,frame=lines]{python}
import numpy as np
import matplotlib.pyplot as plt

# Define parameter sets for two different wells
params_0 = {
    "m_val": 1.0,  # Particle mass
    "V_0_val": 0.0,
    "h_bar_val": 1,
    "a_val": 1.5,  # Small well width
    "b_val": 1.5   # Large well width
}

params_1 = {
    "m_val": 1.0,  # Particle mass
    "V_0_val": 10.0,
    "h_bar_val": 1,
    "a_val": 1.5,  # Small well width
    "b_val": 3.0   # Large well width
}


# Compute energy values for both cases
Energy_0 = Energy_values(**params_0)
Energy_1 = Energy_values(**params_1)

# Quantum numbers corresponding to energy states
n_list_0 = np.arange(1, len(Energy_0) + 1)
n_list_1 = np.arange(1, len(Energy_1) + 1)

# Known energy function for an infinite potential well
E_pozo = lambda n, h_bar, m, b: (n**2 * np.pi**2 * h_bar**2) / (2 * m * (2 * b)**2)

# Create subplots for the two cases
fig, axes = plt.subplots(2, 1, figsize=(12, 8))

# First plot: Well with V_0 = 0
axes[0].scatter(Energy_0, np.zeros_like(Energy_0), color='black', marker='x', label='Numerical Root')
axes[0].scatter(E_pozo(n_list_0, params_0["h_bar_val"], params_0["m_val"], params_0["b_val"]),
                np.zeros_like(n_list_0), color='red', marker='o', label='Infinite Well')
axes[0].set_xlim(0, 40)
axes[0].set_title(f"Energy Levels")
#axes[0].set_xlabel("Energy ")
axes[0].grid()
axes[0].legend()

# Add a text box with parameters
textstr_0 = f"$a = {params_0['a_val']}$\n$b = {params_0['b_val']}$\n$V_0 = {params_0['V_0_val']}$"
props = dict(boxstyle="round,pad=0.3", edgecolor="black", facecolor="white", alpha=0.5)
axes[0].text(0.91, 0.65, textstr_0, transform=axes[0].transAxes, fontsize=12, bbox=props)

# Second plot: Well with V_0 = 10
axes[1].scatter(Energy_1, np.zeros_like(Energy_1), color='black', marker='x', label='Numerical Root')
axes[1].scatter(E_pozo(n_list_1, params_1["h_bar_val"], params_1["m_val"], params_1["b_val"]),
                np.zeros_like(n_list_1), color='red', marker='o', label='Infinite Well')
axes[1].set_xlim(0, 40)
#axes[1].set_title(f"Energy Levels for V_0 = {params_1['V_0_val']}, b = {params_1['b_val']}")
axes[1].set_xlabel("Energy ")
axes[1].grid()
axes[1].legend()

# Add a text box with parameters
textstr_1 = f"$a = {params_1['a_val']}$\n$b = {params_1['b_val']}$\n$V_0 = {params_1['V_0_val']}$"
axes[1].text(0.91, 0.65, textstr_1, transform=axes[1].transAxes, fontsize=12, bbox=props)

# Adjust layout and show the plot
plt.tight_layout()
plt.show()
\end{minted}

\phantomsection
\label{org66d845c}
\begin{verbatim}
<matplotlib.collections.PathCollection at 0x7f73903b3250>
<matplotlib.collections.PathCollection at 0x7f73903b34d0>
\end{verbatim}

\begin{center}
\begin{tabular}{rr}
0.0 & 40.0\\
\end{tabular}
\end{center}
\begin{verbatim}
Text(0.5, 1.0, 'Energy Levels')
<matplotlib.legend.Legend at 0x7f73903b3390>
Text(0.91, 0.65, '$a = 1.5$\n$b = 1.5$\n$V_0 = 0.0$')
<matplotlib.collections.PathCollection at 0x7f7390414cd0>
<matplotlib.collections.PathCollection at 0x7f7390414f50>
\end{verbatim}

\begin{center}
\begin{tabular}{rr}
0.0 & 40.0\\
\end{tabular}
\end{center}
\begin{verbatim}
Text(0.5, 0, 'Energy ')
<matplotlib.legend.Legend at 0x7f7390415090>
Text(0.91, 0.65, '$a = 1.5$\n$b = 3.0$\n$V_0 = 10.0$')
\end{verbatim}

\begin{center}
\includesvg[width=.9\linewidth]{./.ob-jupyter/af8c22d4c7f463a0793394d3574e7150a4917dca}
\end{center}

Como se observa en la gráfica anterior, nuestro cálculo numérico de la
cuantización de la energía coincide con el resultado del pozo infinito
cuando tomamos \(b = a\). Además, al considerar valores \(b > a\), notamos
un desplazamiento en los niveles de energía con respecto al pozo
infinito de ancho \(2a\). En la misma gráfica, también se pueden
identificar los niveles de energía dentro del pozo más pequeño de ancho
\(2a\) y potencial \(V_0\), los cuales resultan ser finitos.
\subsection{\textbf{2.2 Construccion de la funcion de onda}}
\label{sec:org0106a10}
Partimos de las funciones de onda:
\begin{align*}
\psi_0 & = A_0e^{i k_0x} + B_0e^{-i k_0x}, \\
\psi_1 & = A_1e^{i k_1x} + B_1e^{-i k_1x}, \\
\psi_2 & = A_2e^{i k_0x} + B_2e^{-i k_0x}.
\end{align*}

Dado que:
$$ A_0 = - B_0 e^{2ik_0b}, \quad A_2 = - B_2 e^{-2ik_0b},
$$

las funciones de onda pueden reescribirse como:
\begin{align*}
\psi_0 & = -B_0(e^{ik_0 (2b +x)} - e^{-i k_0x}), \\
\psi_1 & = A_1e^{i k_1x} + B_1e^{-i k_1x}, \\
\psi_2 & = -B_2(e^{i k_0(x-2b)} - e^{-i k_0x}).
\end{align*}

Para expresar todas las constantes de integración en términos de
\(B_0\), utilizamos la matriz de transferencia:
$$
\begin{pmatrix} A_1 \\ B_1 \end{pmatrix} = B_2 M_{12}(a)
\begin{pmatrix} e^{-2ik_0b} \\ 1 \end{pmatrix} = B_0 M_{10}(-a)
 \begin{pmatrix} e^{2ik_0b} \\ 1 \end{pmatrix}.
 $$

De esta ecuación se deduce:
$$
 B_2 M_{12}(a) \begin{pmatrix} e^{-2ik_0b} \\ 1 \end{pmatrix} =
 B_0 M_{10}(-a) \begin{pmatrix} e^{2ik_0b} \\ 1 \end{pmatrix}.
 $$

Donde \(M_{10}(-a) = M_{01}(-a)^{-1}\). Lo anterior puede reescribirse como:
$$
\begin{pmatrix} A_1 \\ B_1 \end{pmatrix} =
B_2 \begin{pmatrix} \alpha_1 \\ \beta_1 \end{pmatrix} =
B_0 \begin{pmatrix} \alpha_0 \\ \beta_0 \end{pmatrix}.
 $$

Aquí, \(\alpha_0\), $$ $\beta_0$ y $\beta_1$ son constantes. A partir
de esta relación obtenemos:
$$ B\textsubscript{2} = \frac{\beta_0}{\beta_1}B\textsubscript{0}, \quad A\textsubscript{1} = \(\alpha\)\textsubscript{0}
B\textsubscript{0}, \quad B\textsubscript{1} = \(\beta\)\textsubscript{0} B\textsubscript{0}. \$\$

Finalmente, las funciones de onda quedan expresadas como:
\begin{align*}
\psi_0 &= -B_0(e^{i k_0 (2b+x)} - e^{-ik_0x}), \\
\psi_1 &= B_0(\alpha_0 e^{i k_1x} + \beta_0e^{-ik_1x}), \\
\psi_2 &= -B_0\left(-\frac{\beta_0}{\beta_1} (e^{i k_0(x-2b)} - e^{-ik_0x})\right).
\end{align*}

Procederemos ahora a traducir estas expresiones a un lenguaje simbólico
en Python para encontrar \(\alpha_0\), \(\beta_0\) y \(\beta_1\).

\begin{minted}[breaklines=true,breakanywhere=true,frame=lines]{python}
# Define matrices
M_01 = sym.Matrix([[s01 * sym.exp(i * d01 * a), d01 * sym.exp(i * s01 * a)],
                   [d01 * sym.exp(-i * s01 * a), s01 * sym.exp(-i * d01 * a)]])

M_01 = M_01 / (2 * k_0)

M_12 = sym.Matrix([[s12 * sym.exp(-i * d12 * a), d12 * sym.exp(-i * s12 * a)],
                   [d12 * sym.exp(i * s12 * a), s12 * sym.exp(i * d12 * a)]])

M_12 = M_12 / (2 * k_1)

# inverses
M_10 = M_01.inv()


vector_region_1_0 = M_10 @ sym.Matrix([-sym.exp(2 * i * k_0 * b), 1]) # transfer from 1 to 0
vector_region_1_2 = M_12 @ sym.Matrix([-sym.exp(-2 * i * k_0 * b), 1]) # from 1 to 2

alpha_0 = vector_region_1_0[0].simplify()
beta_0 = vector_region_1_0[1].simplify()

alpha_1 = vector_region_1_2[0].simplify()
beta_1 = vector_region_1_2[1].simplify()
\end{minted}

\begin{minted}[breaklines=true,breakanywhere=true,frame=lines]{python}
alpha_0
\end{minted}

\phantomsection
\label{org44e61dd}
\(\displaystyle \frac{\left(- k_{0} e^{2 i a k_{0}} - k_{0} e^{2 i b k_{0}} + k_{1} e^{2 i a k_{0}} - k_{1} e^{2 i b k_{0}}\right) e^{i a \left(- k_{0} + k_{1}\right)}}{2 k_{1}}\)

\begin{minted}[breaklines=true,breakanywhere=true,frame=lines]{python}
beta_0
\end{minted}

\phantomsection
\label{org6794009}
\(\displaystyle \frac{\left(k_{0} e^{i a \left(3 k_{0} + k_{1}\right)} + k_{1} e^{i a \left(3 k_{0} + k_{1}\right)} + \left(k_{0} - k_{1}\right) e^{i \left(a \left(k_{0} + k_{1}\right) + 2 b k_{0}\right)}\right) e^{- 2 i a \left(k_{0} + k_{1}\right)}}{2 k_{1}}\)

\begin{minted}[breaklines=true,breakanywhere=true,frame=lines]{python}
beta_1
\end{minted}

\phantomsection
\label{org177b504}
\(\displaystyle \frac{\left(\left(k_{0} - k_{1}\right) e^{2 i a k_{0}} + \left(k_{0} + k_{1}\right) e^{2 i b k_{0}}\right) e^{- i \left(a \left(k_{0} - k_{1}\right) + 2 b k_{0}\right)}}{2 k_{1}}\)

Ahora que hemos obtenido las expresiones explícitas de \(\alpha_0\),
\(\beta_0\) y \(\beta_1\), procederemos a calcular numéricamente la
función de onda en cada región y a normalizarla.

\begin{minted}[breaklines=true,breakanywhere=true,frame=lines]{python}
# Symbolic/Numeric Construction of Wavefunctions

# Define the variable for position
x = sym.Symbol('x', real=True)

# Symbolic expressions for the wavefunction in different regions
# Region 0: Outside the well on the left (-b <= x < -a)
phi_0 = sym.exp(i * k_0 * (2 * b + x)) - sym.exp(-i * k_0 * x)

# Region 1: Inside the well (-a <= x <= a)
phi_1 = alpha_0 * sym.exp(i * k_1 * x) + beta_0 * sym.exp(-i * k_1 * x)

# Region 2: Outside the well on the right (a < x <= b)
phi_2 = -(beta_0 / beta_1) * (sym.exp(i * k_0 * (-2 * b + x)) - sym.exp(-i * k_0 * x))

# Substituting k_i values: k_i = sqrt(2 * m * (E - V_i)) / h_bar
phi_0 = phi_0.subs([(k_0, sym.sqrt(2 * m * (E - V_0)) / h_bar), (k_1, sym.sqrt(2 * m * E) / h_bar)])
phi_1 = phi_1.subs([(k_0, sym.sqrt(2 * m * (E - V_0)) / h_bar), (k_1, sym.sqrt(2 * m * E) / h_bar)])
phi_2 = phi_2.subs([(k_0, sym.sqrt(2 * m * (E - V_0)) / h_bar), (k_1, sym.sqrt(2 * m * E) / h_bar)])

# Convert symbolic expressions into numerical functions for fast evaluation
numeric_phi_0 = sym.lambdify((x, E, m, V_0, h_bar, a, b), phi_0, [{'sqrt': np.emath.sqrt}, 'numpy'])
numeric_phi_1 = sym.lambdify((x, E, m, V_0, h_bar, a, b), phi_1, [{'sqrt': np.emath.sqrt}, 'numpy'])
numeric_phi_2 = sym.lambdify((x, E, m, V_0, h_bar, a, b), phi_2, [{'sqrt': np.emath.sqrt}, 'numpy'])

def wavefunction(E: float, m: float, V_0: float, h_bar: float, a: float, b: float):
    """
    Computes the normalized wavefunction for a given energy level in a quantum well.

    Parameters:
    E       : float - Energy level of the wavefunction
    m       : float - Particle mass
    V_0     : float - Potential well depth
    h_bar   : float - Reduced Planck’s constant
    a       : float - Small well width
    b       : float - Large well width

    Returns:
    phi_normalized : function - Normalized wavefunction function
    """

    # Define the squared norm of the wavefunction for numerical integration
    def square_norm(x):
        if -b <= x < -a:
            return np.absolute(numeric_phi_0(x, E, m, V_0, h_bar, a, b)) ** 2
        elif -a <= x <= a:
            return np.absolute(numeric_phi_1(x, E, m, V_0, h_bar, a, b)) ** 2
        elif a < x <= b:
            return np.absolute(numeric_phi_2(x, E, m, V_0, h_bar, a, b)) ** 2
        else:
            return 0  # Zero outside the defined region

    # Perform numerical integration to determine the normalization constant
    norm_integral, _ = quad(square_norm, -b, b)
    normalization_constant = 1 / np.sqrt(norm_integral)

    # Define the normalized wavefunction
    def phi_normalized(x):
        if -b <= x < -a:
            return -numeric_phi_0(x, E, m, V_0, h_bar, a, b) * normalization_constant
        elif -a <= x <= a:
            return numeric_phi_1(x, E, m, V_0, h_bar, a, b) * normalization_constant
        elif a < x <= b:
            return numeric_phi_2(x, E, m, V_0, h_bar, a, b) * normalization_constant
        else:
            return 0  # Zero outside the defined region

    return phi_normalized
\end{minted}

Para ilustrar los resultados, seleccionaremos distintos valores de los
parámetros y generaremos las respectivas gráficas de la función de onda
y su normalización.

\begin{minted}[breaklines=true,breakanywhere=true,frame=lines]{python}
def plot_wavefunction(Energy, m_val, V_0_val, h_bar_val, a_val, b_val):
    """
    Plots the wavefunction for a given energy level in a quantum well system.

    Parameters:
    Energy     : float - Energy level for which the wavefunction is calculated
    m_val      : float - Particle mass
    V_0_val    : float - Potential well depth
    h_bar_val  : float - Reduced Planck’s constant
    a_val      : float - Width of the inner well
    b_val      : float - Width of the outer well

    Returns:
    None (Displays a figure with two subplots: real & imaginary parts and magnitude of the wavefunction)
    """

    # Get the normalized wavefunction for the given parameters
    phi_normalized = wavefunction(Energy, m_val, V_0_val, h_bar_val, a_val, b_val)

    # Define x values for plotting (from -b to b)
    x_values = np.linspace(-b_val, b_val, 1000)

    # Compute the real and imaginary parts of the wavefunction
    phi_values = np.array([np.real(phi_normalized(t)) for t in x_values])
    phi_values_im = np.array([np.imag(phi_normalized(t)) for t in x_values])

    # Compute the magnitude of the wavefunction
    distribution = np.array([np.absolute(phi_normalized(t)) ** 2 for t in x_values])

    # Create a figure with two subplots
    fig, axes = plt.subplots(1, 2, figsize=(15, 5))

    # First subplot: real and imaginary parts of the wavefunction
    axes[0].plot(x_values, phi_values, color='blue', label='Real Part')
    axes[0].plot(x_values, phi_values_im, color='red', linestyle='dashed', label='Imaginary Part')
    axes[0].set_title(f"Wavefunction for Energy = {Energy} eV")
    axes[0].set_xlabel("x")
    axes[0].set_ylabel("Wavefunction Value")
    axes[0].legend()
    axes[0].grid(True)

    # Second subplot: magnitude of the wavefunction
    axes[1].plot(x_values, distribution, color='purple', label='|Ψ(x)|')
    axes[1].set_title(f"Wavefunction for Energy = {Energy} eV")
    axes[1].set_xlabel("x")
    axes[1].set_ylabel("Wavefunction Magnitude")
    axes[1].legend()
    axes[1].grid(True)

    # Adjust layout to prevent overlapping elements and display the plot
    plt.tight_layout()
    plt.show()
\end{minted}

\begin{minted}[breaklines=true,breakanywhere=true,frame=lines]{python}
m_val_0     = 1.0  # Particle mass
V_0_val_0   = 0.0  # Potential well depth
h_bar_val_0 = 1    # Reduced Planck’s constant
a_val_0     = 1.5  # Small well width
b_val_0     = 1.5  # Large well width

Energy_0 = Energy_values(m_val_0, V_0_val_0, h_bar_val_0, a_val_0, b_val_0)
plot_wavefunction(Energy_0[0], m_val_0, V_0_val_0, h_bar_val_0, a_val_0, b_val_0)
plot_wavefunction(Energy_0[4], m_val_0, V_0_val_0, h_bar_val_0, a_val_0, b_val_0)
plot_wavefunction(Energy_0[10], m_val_0, V_0_val_0, h_bar_val_0, a_val_0, b_val_0)
\end{minted}

\phantomsection
\label{org8d5694f}
\begin{center}
\includesvg[width=.9\linewidth]{./.ob-jupyter/cd25d8589f9b677163aad995f6fa5658ca7f2923}
\end{center}
\begin{center}
\includesvg[width=.9\linewidth]{./.ob-jupyter/52e61c3c73fca709c74247c87a2f6ad46a7dd447}
\end{center}
\begin{center}
\includesvg[width=.9\linewidth]{./.ob-jupyter/9dcb30d82670c3a4163da1ac984da10650388ee2}
\end{center}

\begin{minted}[breaklines=true,breakanywhere=true,frame=lines]{python}
m_val_1     = 1.0  # Particle mass
V_0_val_1   = 5.0  # Potential well depth
h_bar_val_1 = 1    # Reduced Planck’s constant
a_val_1     = 1.5  # Small well width
b_val_1     = 10.0 # Large well width

Energy_1 = Energy_values(m_val_1, V_0_val_1, h_bar_val_1, a_val_1, b_val_1)
plot_wavefunction(Energy_1[4], m_val_1, V_0_val_1, h_bar_val_1, a_val_1, b_val_1)
plot_wavefunction(Energy_1[10], m_val_1, V_0_val_1, h_bar_val_1, a_val_1, b_val_1)
plot_wavefunction(Energy_1[15], m_val_1, V_0_val_1, h_bar_val_1, a_val_1, b_val_1)
\end{minted}

\phantomsection
\label{orgef2f894}
\begin{center}
\includesvg[width=.9\linewidth]{./.ob-jupyter/05ab58d1eb9507608415ec63ca922a93aa29fc23}
\end{center}
\begin{center}
\includesvg[width=.9\linewidth]{./.ob-jupyter/4bb71f042a7e47ec840a9fde896b5b369cf41b28}
\end{center}
\begin{center}
\includesvg[width=.9\linewidth]{./.ob-jupyter/0d91b6d39c06e6e0f5b2482bae8a77b97ebc8113}
\end{center}

\begin{minted}[breaklines=true,breakanywhere=true,frame=lines]{python}
m_val_2     = 1.0  # Particle mass
V_0_val_2   = 10.0 # Potential well depth
h_bar_val_2 = 1    # Reduced Planck’s constant
a_val_2     = 1.5  # Small well width
b_val_2     = 20.0 # Large well width

Energy_2 = Energy_values(m_val_2, V_0_val_2, h_bar_val_2, a_val_2, b_val_2)
plot_wavefunction(Energy_2[4], m_val_2, V_0_val_2, h_bar_val_2, a_val_2, b_val_2)
plot_wavefunction(Energy_2[10], m_val_2, V_0_val_2, h_bar_val_2, a_val_2, b_val_2)
plot_wavefunction(Energy_2[30], m_val_2, V_0_val_2, h_bar_val_2, a_val_2, b_val_2)
\end{minted}

\phantomsection
\label{orga089052}
\begin{center}
\includesvg[width=.9\linewidth]{./.ob-jupyter/15bccb7b7c2afdb42447bbdbe31440e190d46fff}
\end{center}
\begin{center}
\includesvg[width=.9\linewidth]{./.ob-jupyter/8ddcdc966beb4ddcd184667bd6392f8a815218b1}
\end{center}
\begin{center}
\includesvg[width=.9\linewidth]{./.ob-jupyter/53a668fbb030df8933bf53294a275266089f8c5e}
\end{center}

Hasta este punto, hemos resuelto un problema de mecánica cuántica. Sin
embargo, nuestro objetivo es incorporar la temperatura en el sistema, lo
que nos lleva a abordar el problema desde la perspectiva de la física
estadística, como exploraremos a continuación.
\subsection{\textbf{2.3 Termalización}}
\label{sec:orgdfb3ee8}
En mecánica cuántica, cuando queremos conocer el comportamiento espacial
de una partícula, utilizamos la densidad de probabilidad posicional,
definida como [3]
$$ P(x,t) = |\Psi(x,t)|^2, $$
donde \(\Psi(x,t)\) es la función de onda correspondiente a la partícula en la
representación de posiciones. Para nuestros propósitos,
considerando energías cuantizadas y con autoespacios no degenerados,
utilizaremos las funciones de onda que corresponden unívocamente a cada nivel de
energía:
$$ P_n(x) = |\Psi_n(x)|^2. $$

Dado que nuestro sistema estará en contacto con un reservorio térmico a
temperatura \(T\), es necesario asignar una densidad de probabilidad canónica al
espacio de medida definido por los niveles de energía del sistema cuántico [4]. El
trabajo de [1] procede a calcular, dada esta densidad canónica, el valor
esperado del observable definido sobre dicho espacio de niveles energéticos
mediante la asignación \(n \mapsto P_n(x)\); la colección de los valores
esperados de estos observables en cada \(x\) conduce a la expresión de
la función densidad de probabilidad termalizada del sistema:
$$
 P_\text{th}(x,T) = \frac{\sum_{n=1}^{\infty}P_n(x)\exp\left[-\frac{E_n}{k_BT}\right]}{\sum_{n=1}^{\infty} \exp[-E_n/k_BT]}.
$$

A continuación, implementamos este procedimiento numéricamente con todos los valores
de energía que obtuvimos de la discusión previa y las temperaturas 1, 10 y 100.

El siguiente es un fichero con todos los métodos que se van a emplear para el cálculo numérico de
las funciones de onda termalizadas (incluye bosones y fermiones, que se
explicarán más adelante).
\begin{minted}[breaklines=true,breakanywhere=true,frame=lines]{python}
%%writefile ../include/functions.h
#ifndef FUNCTIONS_H
#define FUNCTIONS_H

#include <iostream>
#include <fstream>
#include <cmath>
#include <complex>
#include <vector>
#include <tuple>
#include <Eigen/Dense>
#include <gsl/gsl_integration.h>

// Aliases
using Complex   = std::complex<double>;
using MatrixXcd = Eigen::Matrix<Complex, Eigen::Dynamic, Eigen::Dynamic>;
using MatrixXd  = Eigen::Matrix<double, Eigen::Dynamic, Eigen::Dynamic>;
using VectorXd  = Eigen::VectorXd;

// Wave parameters
struct WaveParams {
    double E, m, V_0, h_bar, a, b;
};

// Region wavefunctions
Complex phi_0(double x, const WaveParams& p);
Complex phi_1(double x, const WaveParams& p);
Complex phi_2(double x, const WaveParams& p);

// Normalization
double square_norm(double x, void* params);
double normalize_wavefunction(const WaveParams& p);
Complex phi(double x, const WaveParams& p);

// Wavefunction computations
std::vector<VectorXd> compute_wavefunctions_1d(
    const VectorXd& T_values,
    const VectorXd& energies,
    const VectorXd& x_values,
    double m, double V_0, double h_bar, double a, double b);

std::vector<MatrixXd> compute_wavefunctions(
    const std::vector<std::tuple<double, double, double, double>>& energies,
    const VectorXd& x1_values,
    const VectorXd& x2_values,
    double m, double V_0, double h_bar, double a, double b,
    double T, double sym, bool spin);

// Inpput-output
void save_to_csv_1d(const std::string& filename,
                    const VectorXd& x_values,
                    const VectorXd& T_values,
                    const std::vector<VectorXd>& wavefunctions);

void save_to_csv(const VectorXd& x1_values,
                 const VectorXd& x2_values,
                 const std::vector<std::tuple<double, double, double, double>>& energies,
                 bool spin,
                 const std::vector<MatrixXd>& wavefunctions,
                 const std::string& filename);

#endif
\end{minted}

\phantomsection
\label{orgab06287}
\begin{verbatim}
Overwriting ../include/functions.h
\end{verbatim}

\begin{minted}[breaklines=true,breakanywhere=true,frame=lines]{python}
%%writefile ../src/regions.cpp
#include "../include/functions.h"

// These methods define the wavefunction in each region of the potential well,
// according to quantum mechanical computations.

// Wavefunction over region 0
Complex phi_0(double x, const WaveParams& p) {
    const Complex i(0.0, 1.0);
    const Complex k_0 = (p.E >= p.V_0) ?
        std::sqrt(2.0 * p.m * (p.E - p.V_0)) / p.h_bar :
        i * std::sqrt(2.0 * p.m * (p.V_0 - p.E)) / p.h_bar;
    return std::exp(i * k_0 * (2.0 * p.b + x)) - std::exp(-i * k_0 * x);
}

// Wavefunction over region 1
Complex phi_1(double x, const WaveParams& p) {
    const Complex i(0.0, 1.0);
    const Complex k_1 = std::sqrt(2.0 * p.m * p.E) / p.h_bar;
    const Complex k_0 = (p.E >= p.V_0) ?
        std::sqrt(2.0 * p.m * (p.E - p.V_0)) / p.h_bar :
        i * std::sqrt(2.0 * p.m * (p.V_0 - p.E)) / p.h_bar;

    const Complex alpha_0 = (-k_0 * std::exp(2.0 * i * p.a * k_0) - k_0 * std::exp(2.0 * i * p.b * k_0) +
                            k_1*std::exp(2.0*i*p.a*k_0) - k_1*std::exp(2.0*i*p.b*k_0)) *
                            std::exp(i*p.a*(-k_0 + k_1))/(2.0*k_1);

    const Complex beta_0 = (k_0*std::exp(i*p.a*(3.0*k_0 + k_1)) + k_1*std::exp(i*p.a*(3.0*k_0 + k_1)) +
                           (k_0 - k_1)*std::exp(i*(p.a*(k_0 + k_1) + 2.0*p.b*k_0))) *
                           std::exp(-2.0*i*p.a*(k_0 + k_1))/(2.0*k_1);

    return alpha_0 * std::exp(i * k_1 * x) + beta_0 * std::exp(-i * k_1 * x);
}

// Wavefunction over region 2
Complex phi_2(double x, const WaveParams& p) {
    const Complex i(0.0, 1.0);
    const Complex k_0 = (p.E >= p.V_0) ?
        std::sqrt(2.0 * p.m * (p.E - p.V_0)) / p.h_bar :
        i * std::sqrt(2.0 * p.m * (p.V_0 - p.E)) / p.h_bar;
    const Complex k_1 = std::sqrt(2.0 * p.m * p.E) / p.h_bar;

    const Complex beta_0 = (k_0*std::exp(i*p.a*(3.0*k_0 + k_1)) + k_1*std::exp(i*p.a*(3.0*k_0 + k_1)) +
                           (k_0 - k_1)*std::exp(i*(p.a*(k_0 + k_1) + 2.0*p.b*k_0))) *
                           std::exp(-2.0*i*p.a*(k_0 + k_1))/(2.0*k_1);

    const Complex beta_1 = ((k_0 - k_1)*std::exp(2.0*i*p.a*k_0) + (k_0 + k_1)*std::exp(2.0*i*p.b*k_0)) *
                          std::exp(-i*(p.a*(k_0 - k_1) + 2.0*p.b*k_0))/(2.0*k_1);

    return -(beta_0 / beta_1) * (std::exp(i * k_0 * (-2.0 * p.b + x)) - std::exp(-i * k_0 * x));
}
\end{minted}

\phantomsection
\label{org8504907}
\begin{verbatim}
Overwriting ../src/regions.cpp
\end{verbatim}

\begin{minted}[breaklines=true,breakanywhere=true,frame=lines]{python}
%%writefile ../src/normalisation.cpp
#include "../include/functions.h"

// Wavefunction normalisation regionwise
double square_norm(double x, void* params) {
    auto* p = static_cast<WaveParams*>(params);
    if (x >= -p->b && x < -p->a) {
        return std::norm(phi_0(x, *p));
    } else if (x >= -p->a && x <= p->a) {
        return std::norm(phi_1(x, *p));
    } else if (x > p->a && x <= p->b) {
        return std::norm(phi_2(x, *p));
    }
    return 0.0;
}

// Computes the normalization constant for the wavefunction
double normalize_wavefunction(const WaveParams& p) {
    gsl_integration_workspace* w = gsl_integration_workspace_alloc(1000);
    gsl_function F;
    F.function = &square_norm;
    F.params = const_cast<WaveParams*>(&p);

    double result, error;
    gsl_integration_qag(&F, -p.b, p.b, 0, 1e-6, 1000,
                        GSL_INTEG_GAUSS61, w, &result, &error);
    gsl_integration_workspace_free(w);
    return 1.0 / std::sqrt(result);
}

// Returns the normalized wavefunction value at position x
Complex phi(double x, const WaveParams& p) {
    double norm_const = normalize_wavefunction(p);
    if (x >= -p.b && x < -p.a) {
        return norm_const * phi_0(x, p);
    } else if (x >= -p.a && x <= p.a) {
        return norm_const * phi_1(x, p);
    } else if (x > p.a && x <= p.b) {
        return norm_const * phi_2(x, p);
    }
    return Complex(0.0, 0.0);
}
\end{minted}

\phantomsection
\label{org713379d}
\begin{verbatim}
Overwriting ../src/normalisation.cpp
\end{verbatim}

\begin{minted}[breaklines=true,breakanywhere=true,frame=lines]{python}
%%writefile ../src/wavefunctions.h
#include "../include/functions.h"

// Thermalized probability density function.
// Compute 1-dimensional wavefunctions over a range of temperatures and energies
std::vector<VectorXd> compute_wavefunctions_1d(
    const VectorXd& T_values,
    const VectorXd& energies,
    const VectorXd& x_values,
    double m, double V_0, double h_bar, double a, double b) {

    // Initialise parameters template with E = 0. In time, we will change the
    // energy values
    WaveParams base_params = {0, m, V_0, h_bar, a, b};
    int n_temperatures = T_values.size();
    int n_energies = energies.size();
    int n_x = x_values.size();

    // Declare the result as an (n_temperatures)-dimensional vector of real
    // n_x-dimensional vectors. n_x = amount of x values to compute
    std::vector<VectorXd> result(n_temperatures, VectorXd::Zero(n_x));

    #pragma omp parallel for collapse(1)
    for(int k = 0; k < n_temperatures; ++k) {
        for (int i = 0; i < n_energies; ++i) {
            // overwrite the base params with the correct energy
            WaveParams p = base_params; p.E = energies[i];

            for (int j = 0; j < n_x; ++j) {
                Complex psi = phi(x_values[j], p);
                result[k][j] += exp(-energies[i] / T_values[k]) * std::norm(psi);
            }
        }
    }
    return result;
}

// Returns expected values of the positional probability density
// function for either bosons (sym=1) or fermions (sym=-1) with or without
// spin.
std::vector<MatrixXd> compute_wavefunctions(
    const std::vector<std::tuple<double, double, double, double>>& energies,
    const VectorXd& x1_values,
    const VectorXd& x2_values,
    double m, double V_0, double h_bar, double a, double b,
    double T, double sym, bool spin) {

    // Initialise parameters template with E = 0, in time we will change the
    // energy values
    WaveParams base_params = {0, m, V_0, h_bar, a, b};
    int n_energies = energies.size();
    int n1 = x1_values.size();
    int n2 = x2_values.size();

    // Declare the result as an n_energies-dimensional vector of real
    // (n1xn2)-dimensional matrices
    std::vector<MatrixXd> result(n_energies, MatrixXd::Zero(n1, n2));

    #pragma omp parallel for collapse(1)
    for(int k = 0; k < n_energies; ++k) { // k is the index corresponding to the energy tuple
        // overwrite the base params with the correct energies
        auto [E1, E2, E3, E4] = energies[k];

        WaveParams p1 = base_params; p1.E = E1;
        WaveParams p2 = base_params; p2.E = E2;
        WaveParams p3 = base_params; p3.E = E3;
        WaveParams p4 = base_params; p4.E = E4;

        // populate the k-th array
        for(int i = 0; i < n1; ++i) {
            for(int j = 0; j < n2; ++j) {
                Complex psi1 = phi(x1_values[i], p1);
                Complex psi2 = phi(x2_values[j], p2);
                Complex psi3 = phi(x1_values[i], p3);
                Complex psi4 = phi(x2_values[j], p4);

                if(spin) {
                    result[k](i,j) = 0.5 * std::exp(-(E1 + E2)/T) *
                        (0.25 * std::norm(psi1 * psi2 + psi3 * psi4) +
                         0.75 * std::norm(psi1 * psi2 - psi3 * psi4));
                } else {
                    result[k](i,j) = 0.5 * std::exp(-(E1 + E2)/T) *
                        std::norm(psi1 * psi2 + sym * psi3 * psi4);
                }
            }
        }
    }
    return result;
}
\end{minted}

\phantomsection
\label{org13606c2}
\begin{verbatim}
Overwriting ../src/wavefunctions.h
\end{verbatim}

\begin{minted}[breaklines=true,breakanywhere=true,frame=lines]{python}
%%writefile ../src/input-output.cpp
#include "functions.h"

void save_to_csv_1d(const std::string& filename,
                    const VectorXd& x_values,
                    const VectorXd& T_values,
                    const std::vector<VectorXd>& wavefunctions) {
    std::ofstream file(filename);
    if (!file.is_open()) {
        std::cerr << "Error opening file!" << std::endl;
        return;
    }

    file << "x";
    for (int i = 0; i < T_values.size(); ++i) {
        file << ",T=" << T_values[i];
    }
    file << "\n";

    for (int j = 0; j < x_values.size(); ++j) {
        file << x_values[j];
        for (int k = 0; k < T_values.size(); ++k) {
            file << "," << wavefunctions[k][j];
        }
        file << "\n";
    }
}

void save_to_csv(const VectorXd& x1_values,
                 const VectorXd& x2_values,
                 const std::vector<std::tuple<double, double, double, double>>& energies,
                 bool spin,
                 const std::vector<MatrixXd>& wavefunctions,
                 const std::string& filename) {
    std::ofstream file(filename);

    file << "x1,x2";
    for (const auto& energy_tuple : energies) {
        double E1 = std::get<0>(energy_tuple);
        double E2 = std::get<1>(energy_tuple);
        file << ",(E1=" << E1 << " E2=" << E2 << ")";
    }
    file << "\n";

    for(int i = 0; i < x1_values.size(); ++i) {
        for(int j = 0; j < x2_values.size(); ++j) {
            file << x1_values[i] << "," << x2_values[j];
            for(size_t k = 0; k < wavefunctions.size(); ++k) {
                file << "," << wavefunctions[k](i,j);
            }
            file << "\n";
        }
    }
}
\end{minted}

\phantomsection
\label{org7ade67c}
\begin{verbatim}
Overwriting ../src/input-output.cpp
\end{verbatim}


\begin{minted}[breaklines=true,breakanywhere=true,frame=lines]{python}
%%writefile ../wavefunction_1d.cpp

#include "include/functions.h"
#include <iostream>
#include <fstream>
#include <cmath>
#include <complex>
#include <Eigen/Dense>
#include <gsl/gsl_integration.h>

using Complex = std::complex<double>;
using MatrixXcd = Eigen::Matrix<Complex, Eigen::Dynamic, Eigen::Dynamic>;
using MatrixXd = Eigen::Matrix<double, Eigen::Dynamic, Eigen::Dynamic>;
using VectorXd = Eigen::VectorXd;

// Main implementation for one dimensional plots
int main(int argc, char* argv[]) {
    if (argc < 9) {
        std::cerr << "Usage: " << argv[0] << " m V_0 h_bar a b T sym spin energy_file\n";
        return 1;
    }

    double m     = std::stod(argv[1]);
    double V_0   = std::stod(argv[2]);
    double h_bar = std::stod(argv[3]);
    double a     = std::stod(argv[4]);
    double b     = std::stod(argv[5]);
    double grid_num = std::stod(argv[6]);
    std::string temperature_file  =  argv[7];
    std::string energy_file  =  argv[8];

    std::ifstream T_file(temperature_file);
    VectorXd temperatures;
    double T;
    while (T_file >> T) {
        temperatures.conservativeResize(temperatures.size() + 1);
        temperatures(temperatures.size() - 1) = T;
    }
    T_file.close();

    std::ifstream E_file(energy_file);
    VectorXd energies;
    double E;
    while (E_file >> E) {
        energies.conservativeResize(energies.size() + 1);
        energies(energies.size() - 1) = E;
    }
    E_file.close();

    const int N = grid_num;
    VectorXd x_values = VectorXd::LinSpaced(N, -b, b);

    auto wavefunctions = compute_wavefunctions_1d(temperatures, energies, x_values,
                                                 m, V_0, h_bar, a, b);

    save_to_csv_1d("wavefunctions_1d.csv", x_values, temperatures, wavefunctions);

    return 0;
}
\end{minted}

\phantomsection
\label{org84e936b}
\begin{verbatim}
Overwriting ../wavefunction_1d.cpp
\end{verbatim}

\begin{minted}[breaklines=true,breakanywhere=true,frame=lines]{python}
!cd ../ && make -f Makefile.wavefunction_1d
\end{minted}

\phantomsection
\label{orgffbf6ac}
\begin{verbatim}
g++ -O2 -Wall -std=c++17 -Iinclude -I/usr/include/eigen3 -fopenmp -c src/regions.cpp -o build1d/regions.o
g++ -O2 -Wall -std=c++17 -Iinclude -I/usr/include/eigen3 -fopenmp -c src/normalisation.cpp -o build1d/normalisation.o
g++ -O2 -Wall -std=c++17 -Iinclude -I/usr/include/eigen3 -fopenmp -c src/input-output.cpp -o build1d/input-output.o
g++ -O2 -Wall -std=c++17 -Iinclude -I/usr/include/eigen3 -fopenmp -o bin/wavefunction_1d wavefunction_1d.cpp build1d/regions.o build1d/normalisation.o build1d/wavefunctions.o build1d/input-output.o -lgsl -lgslcblas -lm
\end{verbatim}

A continuación, graficaremos la función \(P_\text{th}\) para el caso de diferentes
pozos y a diferentes temperaturas, evaluando también la capa fronteriza de las
functiones resultantes en \(0.1\), es decir, el valor de \(x\) en el que asumen
dicho valor; esta capa fronteriza es una medida del nivel de homogeinización de
la función, puesto que entre más pequeño sea, más se acerca la función de onda a
la pared del pozo.
\begin{minted}[breaklines=true,breakanywhere=true,frame=lines]{python}
import subprocess

def wavefunction_1d(m_val, V_0_val, h_bar_val, a_val, b_val, grid_num=1000, n_energy=40):
    # Directories
    BIN_DIR = "../bin"
    WAVE_EXE = f"{BIN_DIR}/wavefunction_1d"
    WAVE_CSV = f"wavefunctions_1d.csv"
    ENERGIES_FILE = f"{BIN_DIR}/energies.txt"
    TEMPS_FILE = f"{BIN_DIR}/temperatures.txt"

    # Calculate energy values (assuming Energy_values is a predefined function)
    E_list = np.array(Energy_values(m_val, V_0_val, h_bar_val, a_val, b_val))[:n_energy]
    T_list = np.array([1, 10, 50, 100, 150, 200, 250, 300])

    # Write energies to a text file
    with open(ENERGIES_FILE, "w") as f:
        for E in E_list:
            f.write(f"{E}\n")

    # Write temperatures to a text file
    with open(TEMPS_FILE, "w") as f:
        for T in T_list:
            f.write(f"{T}\n")

    # Execute the external program wavefunction_1d
    try:
        print("Running wavefunction_1d...")
        run_command = [
            WAVE_EXE, str(m_val), str(V_0_val), str(h_bar_val),
            str(a_val), str(b_val), str(grid_num), TEMPS_FILE, ENERGIES_FILE
        ]
        subprocess.run(run_command, check=True)
        print("Execution completed.")
    except subprocess.CalledProcessError:
        print("Error during execution.")
        return

    # Read wavefunction data from CSV
    wave_data_1d = pd.read_csv(WAVE_CSV)

    # Plot the boundary layer at 0.1
    boundary_layer = []
    x_values = wave_data_1d.iloc[:, 0]
    for i in range(1, len(T_list) + 1):
        Pth_values = wave_data_1d.iloc[:, i]
        tol = np.abs(np.diff(Pth_values.to_numpy()).min()) / 2
        vals = x_values[(Pth_values.between(0.1 - tol, 0.1 + tol)) & (x_values > 0)]
        mean_positive_vals = vals.mean()
        boundary_layer.append(b_val - mean_positive_vals)

    plt.plot(T_list, boundary_layer, '-o')
    plt.xlabel('T')
    plt.ylabel('Boundary layer (d)')
    plt.show()

    # Plot the wavefunctions
    columns = wave_data_1d.columns
    plt.figure(figsize=(10, 6))
    for i in range(1, 5):
        wave_data_1d[columns[i]] = wave_data_1d[columns[i]] / np.exp(-E_list / T_list[i - 1]).sum()
        plt.plot(wave_data_1d['x'], wave_data_1d[columns[i]], label=f"T={T_list[i - 1]}")

    plt.xlabel("x")
    plt.ylabel("Pth")
    plt.legend()
    plt.show()
\end{minted}

\begin{minted}[breaklines=true,breakanywhere=true,frame=lines]{python}
m_val     = 1.0  # Particle mass
V_0_val   = 10.0 # Potential well depth
h_bar_val = 1    # Reduced Planck’s constant
a_val     = 1.5  # Small well width
b_val     = 1.5  # Large well width
grid_num  = 1000
n_energy = 60

wavefunction_1d(m_val, V_0_val, h_bar_val, a_val, b_val,grid_num)
\end{minted}

\phantomsection
\label{org9fa4c20}
\begin{verbatim}
Running wavefunction_1d...
Execution completed.
\end{verbatim}

\begin{center}
\includesvg[width=.9\linewidth]{./.ob-jupyter/d17e53ebf6a8c89de8a4b76c2b05cd03449e0264}
\end{center}
\begin{center}
\includesvg[width=.9\linewidth]{./.ob-jupyter/acbc1c6c9bbe0127eb96a10ebe501fddf427e3be}
\end{center}

\begin{minted}[breaklines=true,breakanywhere=true,frame=lines]{python}
m_val     = 1.0  # Particle mass
V_0_val   = 10.0 # Potential well depth
h_bar_val = 1    # Reduced Planck’s constant
a_val     = 1.5  # Small well width
b_val     = 6    # Large well width
grid_num  = 1000
n_energy  = 70

wavefunction_1d(m_val, V_0_val, h_bar_val, a_val, b_val,grid_num)
\end{minted}

\phantomsection
\label{org979e308}
\begin{verbatim}
Running wavefunction_1d...
Execution completed.
\end{verbatim}

\begin{center}
\includesvg[width=.9\linewidth]{./.ob-jupyter/44609e750fda618e6948c3b037476d5ddf97605f}
\end{center}
\begin{center}
\includesvg[width=.9\linewidth]{./.ob-jupyter/e002b4d9824ff899e89404c59801834007ad34cc}
\end{center}

\begin{minted}[breaklines=true,breakanywhere=true,frame=lines]{python}
m_val     = 1.0  # Particle mass
V_0_val   = 10.0  # Potential well depth
h_bar_val = 1    # Reduced Planck’s constant
a_val     = 1.5  # Small well width
b_val     = 15   # Large well width
grid_num  = 1000
n_energy  = 60

wavefunction_1d(m_val, V_0_val, h_bar_val, a_val, b_val,grid_num)
\end{minted}

\phantomsection
\label{orgbdcb23c}
\begin{verbatim}
Running wavefunction_1d...
Execution completed.
\end{verbatim}

\begin{center}
\includesvg[width=.9\linewidth]{./.ob-jupyter/084ba81b2ceedc49c68bdde5c7fc58301cc90915}
\end{center}
\begin{center}
\includesvg[width=.9\linewidth]{./.ob-jupyter/3591cdf782a1090dc9cdc87b73997c83dd6829ef}
\end{center}

\begin{minted}[breaklines=true,breakanywhere=true,frame=lines]{python}
m_val     = 1.0  # Particle mass
V_0_val   = 10.0  # Potential well depth
h_bar_val = 1    # Reduced Planck’s constant
a_val     = 1.5  # Small well width
b_val     = 30   # Large well width
grid_num  = 1000
n_energy  = 60

wavefunction_1d(m_val, V_0_val, h_bar_val, a_val, b_val,grid_num)
\end{minted}

\phantomsection
\label{org7ef58a2}
\begin{verbatim}
Running wavefunction_1d...
Execution completed.
\end{verbatim}

\begin{center}
\includesvg[width=.9\linewidth]{./.ob-jupyter/ba7ceb9bdd429167a7bbbcb9ec7241d12cc83ece}
\end{center}
\begin{center}
\includesvg[width=.9\linewidth]{./.ob-jupyter/1db30433ee54cbe4d6a6e29c0b50306d1ce926ea}
\end{center}

\noindent\rule{\textwidth}{0.5pt}
\section{Bosones y fermiones con y sin espín}
\label{sec:orgb01af17}

Cuando consideramos sistemas de \(N\) partículas cuánticas, es necesario tener
en cuenta dos principios físicos que pueden regir al sitema dado:
Distinguibilidad y principio de exclusión en niveles energéticos; estas
propiedades se ponen de manifiesto en ciertas construcciones algebraicas, las
cuales se resumen a continuación.

Para empezar, es necesario entender qué espacios vectoriales representan a un
sistema de \(N\) partículas cuánticas desde el punto de vista de la mecánica
estadística. El espacio de medida subyacente a tal sistema estará dado por el
conjunto de tuplas de números naturales \((n_1, n_2, n_3, ...) \in
\mathbb{N}[\![X]\!]\), tales que \(n_1 + n_2 + n_3 + ... = N\) {[}4];
notemos que todas las tuplas que satisfagan esta relación deben ser casi nulas (i.e., todas
sus entradas, salvo un número finito de ellas, son 0), por lo cual las
expresiones siguientes estarán bien definidas. Estos \(n_i\) representan el
número de ocupación del nivel energético \$i\$-ésimo, que viene determinado por
el autoespacio \(\mathcal{H}_i\) del autoestado \(E_i\) del Hamiltoniano.

La medida correspondiente a cada tupla de este espacio de medida estará dada por
la dimensión del espacio vectorial siguiente:
$$
\text{dim } \big( \text{prod}^{n_1} \, \mathcal{H}_1 \otimes \text{prod}^{n_2} \,
\mathcal{H}_2 \otimes \text{prod}^{n_3} \, \mathcal{H}_3 \otimes ... \big) =
\prod_{i=1}^\infty \text{dim } \text{prod}^{n_i} \, \mathcal{H}_i
$$
donde \(\text{prod}\) denota alguna de las construcciones algebraicas derivadas
del producto tensorial y que dependen de la naturaleza física del problema en
cuestión; a saber, estos productos deberán codificar la información sobre la
distinguibilidad y la exclusión de las \(n_i\) partículas en el nivel \(E_i\).

Los productos posibles que se mencionan en el párrafo anterior son los
siguientes [3]:
\begin{itemize}
\item Partículas clásicas: Distinguibles, no tienen exclusión en niveles
energéticos. En este caso, la construcción algebraica es el producto tensorial
usual; si \(n\) es un número de ocupación del autoespacio \(\mathcal{H}\),
entonces se considera \(\bigotimes^n \mathcal{H}\), cuya dimensión es
\((\text{dim } \mathcal{H})^n\). La razón es que la distinguibilidad es
capturada por la no conmutatividad del producto tensorial, \(v_1 \otimes v_2
  \neq v_2 \otimes v_1\); mientras que el principio de exclusión no se satisface
en la medida en que \(v \otimes v \neq 0\) en general.

\item Fermiones (cuánticos): Indistinguibles, tienen exclusión en niveles
energéticos. En este caso, la construcción algebraica es el producto cuña; si
\(n\) es un número de ocupación del autoespacio \(\mathcal{H}\), entonces se
considera \(\bigwedge^n \mathcal{H}\), cuya dimensión es \(\text{dim }
  \mathcal{H} \choose n\). La razón es que la indistinguibilidad es capturada
por la antisimetría del producto tensorial, \(v_1 \wedge v_2 = -v_2 \wedge
  v_1\), el signo es irrelevante a nivel cuántico porque se consideran estados
normalizados (las proyecciones en el espacio proyectivo); mientras que el
principio de exclusión se satisface en la medida en que \(v \wedge v = 0\).

\item Bosones (cuánticos): Indistinguibles, no tienen exclusión en niveles
energéticos. En este caso, la construcción algebraica es el producto simétrico; si
\(n\) es un número de ocupación del autoespacio \(\mathcal{H}\), entonces se
considera \(\bigodot^n \mathcal{H}\), cuya dimensión es \(\text{dim }
  \mathcal{H} + n -1 \choose n\). La razón es que la indistinguibilidad es capturada
por la simetría del producto tensorial, \(v_1 \odot v_2 = v_2 \odot
  v_1\); mientras que el
principio de exclusión no se satisface en la medida en que \(v \odot v = 0\)
en general.
\end{itemize}

Cuando se consideran estos sistemas, lo usual es proyectar el producto tensorial
de las funciones de onda a los productos simétrico o antisimétrico por medio de
las operaciones conocidas como simetrizador y antisimetrizador,
correspondientemente; estas son, en el caso de dos funciones de onda:
\begin{itemize}
\item \textbf{Simetrizador (bosones)}:
$$
  S(\psi_1 \otimes \psi_2) = \frac{1}{2} (\psi_1 \otimes \psi_2 + \psi_2 \otimes \psi_1)
  $$
Esta proyección garantiza que el estado resultante sea simétrico bajo la
permutación de partículas, lo que es característico de los bosones.

\item \textbf{Antisimetrizador (fermiones)}:
$$
  A(\psi_1 \otimes \psi_2) = \frac{1}{2} (\psi_1 \otimes \psi_2 - \psi_2 \otimes \psi_1)
  $$
Esta operación impone la antisimetría del estado.
\end{itemize}

En general, se tiene que
\begin{itemize}
\item El operador simetrizador para bosones:
$$
  S = \frac{1}{N!} \sum_{\sigma \in S_N} P_\sigma
  $$
\item El operador antisimetrizador para fermiones:
$$
  A = \frac{1}{N!} \sum_{\sigma \in S_N} \text{sgn}(\sigma) P_\sigma
  $$
\end{itemize}
donde \$ P\textsubscript{\(\sigma\)} \$ es el operador de permutación asociado a \$ \(\sigma\) \$ y
\$ \text{sgn}(\(\sigma\)) \$ es la paridad de la permutación (\(+1\) para
permutaciones pares y \(-1\) para impares).

Con estas expresiones se construyen las nuevas funciones de onda en los casos
correspondientes, y se calcula la densidad de probabilidad \(P\) de manera análoga al caso de una sola partícula.

El espín en los fermiones se considera de manera más heurística, a saber, se
impone que la función de onda puede ser antisimétrica en posiciones o en espines
con la combinación total de ambas antisimétrica bajo el intercambio de
partículas. Esto se debe al principio de exclusión.

\begin{itemize}
\item \textbf{\textbf{Función de onda espacial simétrica}}:
 $$
   \Psi_{\text{sf}, n_1 n_2} = \frac{1}{\sqrt{2}} \left[ \varphi_{n_1}(x_1) \varphi_{n_2}(x_2) + \varphi_{n_1}(x_2) \varphi_{n_2}(x_1) \right] \frac{1}{\sqrt{2}} \left[ \chi_+(1) \chi_-(2) - \chi_+(2) \chi_-(1) \right]
   $$
En este caso, la parte espacial de la función de onda es simétrica bajo la
permutación de \(x_1\) y \(x_2\), lo que implica que la parte de espín debe
ser antisimétrica para conservar la antisimetría total.

\item \textbf{\textbf{Función de onda espacial antisimétrica}}:
$$
   \Psi_{\text{af}, n_1 n_2} = \frac{1}{\sqrt{2}} \left[ \varphi_{n_1}(x_1)
   \varphi_{n_2}(x_2) - \varphi_{n_1}(x_2) \varphi_{n_2}(x_1) \right]
   \frac{1}{\sqrt{2}} \left[ \chi_+(1) \chi_-(2) - \chi_-(1) \chi_+(2) \right]
   $$
$$
   \Psi_{\text{af}, n_1 n_2} = \frac{1}{\sqrt{2}} \left[ \varphi_{n_1}(x_1) \varphi_{n_2}(x_2) - \varphi_{n_1}(x_2) \varphi_{n_2}(x_1) \right] \frac{1}{\sqrt{2}} \left[ \chi_+(1) \chi_+(2) \right]
   $$
$$
   \Psi_{\text{af}, n_1 n_2} = \frac{1}{\sqrt{2}} \left[ \varphi_{n_1}(x_1) \varphi_{n_2}(x_2) - \varphi_{n_1}(x_2) \varphi_{n_2}(x_1) \right] \frac{1}{\sqrt{2}} \left[ \chi_-(1) \chi_-(2) \right]
   $$
En estos casos, la parte espacial es antisimétrica bajo el intercambio de \$
x\textsubscript{1} \$ y \$ x\textsubscript{2} \$, por lo que la parte de espín debe ser simétrica.
\end{itemize}

De este modo, la función de onda total siempre se mantiene antisimétrica,
asegurando que el sistema obedezca las reglas impuestas por la estadística de
Fermi-Dirac. Según estas expresiones, al calcular la densidad de probabilidad
\(P\), hay 3 partes aportadas por la función de onda antisimétrica y 1 parte
aportada por la simétrica, por lo que se emplea la expresión:
$$
P_{n_1, n_2} (x_1, x_2) = \frac{1}{4} \left( \frac{1}{2} \left[
\varphi_{n_1}(x_1) \varphi_{n_2}(x_2) + \varphi_{n_1}(x_2) \varphi_{n_2}(x_1)
\right]^2 \right) + \frac{3}{4} \left( \frac{1}{2} \left[ \varphi_{n_1}(x_1) \varphi_{n_2}(x_2) - \varphi_{n_1}(x_2) \varphi_{n_2}(x_1) \right]^2 \right).
$$

Estas expresiones fueron consideradas en el código de C++ para sistemas de dos
partículas, y se implementaron en el siguiente bloque de código:
\begin{minted}[breaklines=true,breakanywhere=true,frame=lines]{python}
%%writefile ../wavefunction.cpp

#include "include/functions.h"
#include <iostream>
#include <fstream>
#include <cmath>
#include <complex>
#include <Eigen/Dense>
#include <gsl/gsl_integration.h>
#include <tuple>
#include <vector>

using Complex = std::complex<double>;
using MatrixXcd = Eigen::Matrix<Complex, Eigen::Dynamic, Eigen::Dynamic>;
using MatrixXd = Eigen::Matrix<double, Eigen::Dynamic, Eigen::Dynamic>;
using VectorXd = Eigen::VectorXd;

// Main implementation for surface plots
int main(int argc, char* argv[]) {
    if (argc < 11) {
        std::cerr << "Usage: " << argv[0] << " m V_0 h_bar a b T sym spin energy_file\n";
        return 1;
    }

    double m     = std::stod(argv[1]);
    double V_0   = std::stod(argv[2]);
    double h_bar = std::stod(argv[3]);
    double a     = std::stod(argv[4]);
    double b     = std::stod(argv[5]);
    double T     = std::stod(argv[6]);
    double sym   = std::stod(argv[7]);
    bool spin    = std::stod(argv[8]);
    double grid_num = std::stod(argv[9]);
    std::string energy_file  =  argv[10];

    std::ifstream file(energy_file);
    std::vector<std::tuple<double, double, double, double>> energies;
    double E1, E2, E3, E4;
    while (file >> E1 >> E2 >> E3 >> E4) {
        energies.emplace_back(E1, E2, E3, E4);
    }
    file.close();

    const int N = grid_num;
    VectorXd x1_values = VectorXd::LinSpaced(N, -b, b);
    VectorXd x2_values = x1_values.array() + 0.001;

    auto wavefunctions = compute_wavefunctions(energies, x1_values, x2_values,
                                            m, V_0, h_bar, a, b, T, sym, spin);

    save_to_csv(x1_values, x2_values, energies, spin, wavefunctions, "wavefunctions.csv");

    return 0;
}
\end{minted}

\phantomsection
\label{org5d00081}
\begin{verbatim}
Overwriting ../wavefunction.cpp
\end{verbatim}

\begin{minted}[breaklines=true,breakanywhere=true,frame=lines]{python}
!cd ../ && make -f Makefile.wavefunction
\end{minted}

\phantomsection
\label{orga623997}
\begin{verbatim}
g++ -O2 -Wall -std=c++17 -Iinclude -I/usr/include/eigen3 -fopenmp -c src/regions.cpp -o build/regions.o
g++ -O2 -Wall -std=c++17 -Iinclude -I/usr/include/eigen3 -fopenmp -c src/normalisation.cpp -o build/normalisation.o
g++ -O2 -Wall -std=c++17 -Iinclude -I/usr/include/eigen3 -fopenmp -c src/input-output.cpp -o build/input-output.o
g++ -O2 -Wall -std=c++17 -Iinclude -I/usr/include/eigen3 -fopenmp -o bin/wavefunction wavefunction.cpp build/regions.o build/normalisation.o build/wavefunctions.o build/input-output.o -lgsl -lgslcblas -lm
\end{verbatim}

A continuación, mostraremos un ejemplo de la implementación del código anterior
para un sistema de dos fermiones en una caja infinita y restringidas a ocupar
tan solo dos niveles de energía.
\begin{minted}[breaklines=true,breakanywhere=true,frame=lines]{python}
BIN_DIR = f"../bin"
WAVE_EXE = f"{BIN_DIR}/wavefunction"

# Define parameter sets
m_val     = 1.0  # Particle mass
V_0_val   = 1.0  # Potential well depth
h_bar_val = 1    # Reduced Planck’s constant
a_val     = 1.5  # Small well width
b_val     = 1.5  # Large well width
sym       = -1   # Antisymmetric function, i.e., we consider fermions
spin      = 0
grid_num  = 50
T         = 10

Energy = Energy_values(m_val, V_0_val, h_bar_val, a_val, b_val )

# We only consider two of the energy levels to be occupied by the fermions!
l = [Energy_0[0], Energy_0[4]]

# Combinations with replacement for energies are considered in the case of fermions
tuples = list(combinations_with_replacement(l, 2))

new_tuples = [(a, b, b, a) for a, b in tuples[1:-1]]

# Write to a text file
with open("tuples.txt", "w") as f:
    for t in new_tuples:
        f.write(f"{t[0]} {t[1]} {t[2]} {t[3]} \n")

print("Tuples saved to tuples.txt")

try:
    print("Ejecutando wavefunction...")
    run_command = [WAVE_EXE, str(m_val), str(V_0_val), str(h_bar_val), str(a_val), str(b_val), str(T), str(sym), str(spin),str(grid_num), "tuples.txt"]
    subprocess.run(run_command, check=True)
    print("Ejecución completada.")
except subprocess.CalledProcessError:
    print("Error en la ejecución.")
    exit(1)
\end{minted}

\phantomsection
\label{org9975608}
\begin{verbatim}
77Tuples saved to tuples.txt
Ejecutando wavefunction...
CompletedProcess(args=['../bin/wavefunction', '1.0', '1.0', '1', '1.5', '1.5', '10', '-1', '0', '50', 'tuples.txt'], returncode=0)Ejecución completada.
\end{verbatim}

\begin{minted}[breaklines=true,breakanywhere=true,frame=lines]{python}
wave_data = pd.read_csv('wavefunctions.csv')
wave_data["Pth_f"] = wave_data.iloc[:,2:2+len(tuples)].sum(axis=1) / np.exp(-np.array(tuples).sum(axis = 1)/T).sum()
print_table(wave_data.head(10))
\end{minted}

\phantomsection
\label{orga9bc4df}
\begin{verbatim}
|    |   x1 |       x2 |   (E1=0.548311 E2=13.7078) |       Pth_f |
|----+------+----------+----------------------------+-------------|
|  0 | -1.5 | -1.499   |                9.65042e-38 | 8.03553e-38 |
|  1 | -1.5 | -1.43778 |                3.66729e-34 | 3.05361e-34 |
|  2 | -1.5 | -1.37655 |                1.36782e-33 | 1.13893e-33 |
|  3 | -1.5 | -1.31533 |                2.81059e-33 | 2.34027e-33 |
|  4 | -1.5 | -1.2541  |                4.46309e-33 | 3.71624e-33 |
|  5 | -1.5 | -1.19288 |                6.14715e-33 | 5.11849e-33 |
|  6 | -1.5 | -1.13165 |                7.80864e-33 | 6.50195e-33 |
|  7 | -1.5 | -1.07043 |                9.53594e-33 | 7.94021e-33 |
|  8 | -1.5 | -1.0092  |                1.15179e-32 | 9.59051e-33 |
|  9 | -1.5 | -0.94798 |                1.39572e-32 | 1.16216e-32 |
\end{verbatim}

En la tabla que hemos construido, se presentan valores de \(x_1\) y \(x_2\)
donde se evalúan los términos del numerador de la expresión \(P_{th}\) para los valores de
energía que se tomaron; estos términos luego se suman y dividen por la
normalización para obtener \(P_{th}\), que es la última columna.

\begin{minted}[breaklines=true,breakanywhere=true,frame=lines]{python}
#Grafiacar la función Pth_f
X1 = wave_data['x1'].to_numpy().reshape((grid_num,grid_num))
X2 = wave_data['x2'].to_numpy().reshape((grid_num,grid_num))
Z  = wave_data['Pth_f'].to_numpy().reshape((grid_num,grid_num))

# Configurar la figura 3D
fig = plt.figure(figsize=(8,8))
ax = fig.add_subplot(111, projection='3d')

# Dibujar la superficie
surf = ax.plot_surface(X1, X2, Z, cmap="turbo", edgecolor="k", linewidth=0.1, alpha=1)

# Etiquetas de ejes
ax.set_xlabel("x1")
ax.set_ylabel("x2")
ax.set_zlabel("P")

#Eliminar fondo
ax.xaxis.pane.fill = False
ax.yaxis.pane.fill = False
ax.zaxis.pane.fill = False

# quitar cuadrícula
ax.grid(False)

#Demarcar bien los ejes
ax.xaxis.pane.set_edgecolor('w')
ax.yaxis.pane.set_edgecolor('w')
ax.zaxis.pane.set_edgecolor('w')

# Configurar los límites de la caja
ax.set_xlim(-b_val, b_val)
ax.set_ylim(-b_val, b_val)
ax.set_zlim(0, Z.max())

# Ajustar la relación de aspecto para ensanchar
ax.set_box_aspect([2, 2, 1])

# Configurar la vista
ax.view_init(elev=45, azim=45)

# Mostrar gráfico
plt.show()
\end{minted}

\phantomsection
\label{org85fee94}
\begin{verbatim}
Text(0.5, 0, 'x1')
Text(0.5, 0.5, 'x2')
Text(0.5, 0, 'P')
\end{verbatim}

\begin{center}
\begin{tabular}{rr}
-1.5 & 1.5\\
-1.5 & 1.5\\
0.0 & 0.14493967076642433\\
\end{tabular}
\end{center}
\begin{center}
\includesvg[width=.9\linewidth]{./.ob-jupyter/0b7a554b6bf48b8c03bc2cc3d819af65d71af806}
\end{center}

Esta es la superficie resultante de la función \(P_{th}\) asociada a los niveles
\(E_0\) y \(E_4\) en los valores de \(x_1\) y \(x_2\).

\noindent\rule{\textwidth}{0.5pt}
\section{\textbf{3. Resultados y discusión}}
\label{sec:orgf274049}
\subsection{Código:}
\label{sec:org318b16f}
Ahora procedemos a sistematizar el proceso, para considerar la mayor cantidad de
casos posible.
\begin{minted}[breaklines=true,breakanywhere=true,frame=lines]{python}
def papers(E, m_val, V_0_val, h_bar_val, a_val, b_val, T, sym, spin, grid_num=60):
    """
    Generates and visualizes the thermal wavefunction in 3D.

    Parameters:
        E (list): Indices of the energy levels to consider.
        m_val (float): Particle mass.
        V_0_val (float): Potential barrier value.
        h_bar_val (float): Reduced Planck's constant.
        a_val (float): Barrier parameter (central region width).
        b_val (float): Barrier parameter (total extent).
        T (float): System temperature.
        sym (int): Specifies whether symmetric energy combinations (bosons/fermions) are considered.
        spin (int): Specifies whether spin coupling is considered.
        grid_num (int, optional): Grid size for visualization (default = 60).
    """

    # Obtain the energy values for the given parameters
    Energy = Energy_values(m_val, V_0_val, h_bar_val, a_val, b_val)

    # Check if symmetry is applied to energy combinations
    if sym == 1 and len(E) > 2:
        # Select energy values using the indices in E
        l = [Energy[i] for i in E]
        # Generate combinations with replacement (considering symmetry)
        tuples = list(combinations_with_replacement(l, 2))
        # Create new tuples formatted as (a, b, b, a)
        new_tuples = [(a, b, b, a) for a, b in tuples]

    else:
        # Select energy values using the indices in E
        l = [Energy[i] for i in E]
        # Generate permutations without replacement (no symmetry)
        tuples = list(itertools.permutations(l, 2))
        # Create new tuples formatted as (a, b, b, a)
        new_tuples = [(a, b, b, a) for a, b in tuples]

    # Save the energy combinations to a text file
    with open("tuples.txt", "w") as f:
        for t in new_tuples:
            f.write(f"{t[0]} {t[1]} {t[2]} {t[3]}\n")

    # Execute the C++ program that calculates the wavefunctions
    try:
        print("Running wavefunction...")
        run_command = [
            WAVE_EXE, str(m_val), str(V_0_val), str(h_bar_val),
            str(a_val), str(b_val), str(T), str(sym), str(spin),
            str(grid_num), "tuples.txt"
        ]
        subprocess.run(run_command, check=True)
        print("Execution completed.")
    except subprocess.CalledProcessError:
        print("Execution error.")
        exit(1)

    # Load the data generated by the C++ program
    wave_data = pd.read_csv('wavefunctions.csv')

    # Compute the normalized thermal probability
    wave_data["Pth_f"] = wave_data.iloc[:, 2:2 + len(tuples)].sum(axis=1) / np.exp(-np.array(tuples).sum(axis=1) / T).sum()

    # Extract spatial coordinates and thermal distribution
    X1 = wave_data['x1'].to_numpy().reshape((grid_num, grid_num))
    X2 = wave_data['x2'].to_numpy().reshape((grid_num, grid_num))
    Z = wave_data['Pth_f'].to_numpy().reshape((grid_num, grid_num))

    # Configure the 3D figure
    fig = plt.figure(figsize=(8, 8))
    ax = fig.add_subplot(111, projection='3d')

    # Plot the 3D surface with the "turbo" colormap
    surf = ax.plot_surface(X1, X2, Z, cmap="turbo", edgecolor="k", linewidth=0.1, alpha=1)

    # Axis labels
    ax.set_xlabel("x1")
    ax.set_ylabel("x2")
    ax.set_zlabel("P")

    # Remove background from the axes
    ax.xaxis.pane.fill = False
    ax.yaxis.pane.fill = False
    ax.zaxis.pane.fill = False

    # Remove grid for better visualization
    ax.grid(False)

    # Adjust axis borders
    ax.xaxis.pane.set_edgecolor('w')
    ax.yaxis.pane.set_edgecolor('w')
    ax.zaxis.pane.set_edgecolor('w')

    # Set plot limits
    ax.set_xlim(-b_val, b_val)
    ax.set_ylim(-b_val, b_val)
    ax.set_zlim(0, Z.max())

    # Adjust the box aspect ratio for better visualization
    ax.set_box_aspect([2, 2, 1])

    # Set the 3D view
    ax.view_init(elev=45, azim=45)

    # Show the figure
    plt.show()
\end{minted}

\noindent\rule{\textwidth}{0.5pt}
Primero reproduciremos los resultados del paper [1], en el caso partícular en
el que \(a = b\).
\begin{minted}[breaklines=true,breakanywhere=true,frame=lines]{python}
papers(
    E         = [0,0],
    m_val     = 1.0,
    V_0_val   = 10.0,
    h_bar_val = 1.0,
    a_val     = 1.5,
    b_val     = 1.5,
    T         = 1,
    sym       = 1,
    spin      = 0,
    grid_num  = 60
)
\end{minted}

\phantomsection
\label{org7c90040}
\begin{verbatim}
Running wavefunction...
Execution completed.
\end{verbatim}

\begin{center}
\includesvg[width=.9\linewidth]{./.ob-jupyter/e41b31f2428a68b1d8c2a8bd345fa043f3e32e96}
\end{center}

\begin{minted}[breaklines=true,breakanywhere=true,frame=lines]{python}
papers(
    E         = [0,4],
    m_val     = 1.0,
    V_0_val   = 10.0,
    h_bar_val = 1.0,
    a_val     = 1.5,
    b_val     = 1.5,
    T         = 1,
    sym       = 1,
    spin      = 0,
    grid_num  = 60
)
\end{minted}

\phantomsection
\label{org8b46671}
\begin{verbatim}
Running wavefunction...
Execution completed.
\end{verbatim}

\begin{center}
\includesvg[width=.9\linewidth]{./.ob-jupyter/ef8630dca962eee0eeffb8460e11b28d39f87463}
\end{center}

\begin{minted}[breaklines=true,breakanywhere=true,frame=lines]{python}
papers(
    E         = [0,1],
    m_val     = 1.0,
    V_0_val   = 10.0,
    h_bar_val = 1.0,
    a_val     = 1.5,
    b_val     = 1.5,
    T         = 1,
    sym       = -1,
    spin      = 0,
    grid_num  = 60
)
\end{minted}

\phantomsection
\label{org56aa845}
\begin{verbatim}
Running wavefunction...
Execution completed.
\end{verbatim}

\begin{center}
\includesvg[width=.9\linewidth]{./.ob-jupyter/a6ae65f5108b4e9bd9f38b4bd10c71a25f816386}
\end{center}

\begin{minted}[breaklines=true,breakanywhere=true,frame=lines]{python}
papers(
    E         = [0,4],
    m_val     = 1.0,
    V_0_val   = 10.0,
    h_bar_val = 1.0,
    a_val     = 1.5,
    b_val     = 1.5,
    T         = 1,
    sym       = -1,
    spin      = 0,
    grid_num  = 60
)
\end{minted}

\phantomsection
\label{orgee97f2c}
\begin{verbatim}
Running wavefunction...
Execution completed.
\end{verbatim}

\begin{center}
\includesvg[width=.9\linewidth]{./.ob-jupyter/52888101a7a0d09287d7d5772aa65ab8667372c5}
\end{center}

\begin{minted}[breaklines=true,breakanywhere=true,frame=lines]{python}
papers(
    E         = [i for i in range(10)],
    m_val     = 1.0,
    V_0_val   = 0.0,
    h_bar_val = 1.0,
    a_val     = 1.5,
    b_val     = 1.5,
    T         = 10,
    sym       =-1,
    spin      = 0,
    grid_num  = 60
)
\end{minted}

\phantomsection
\label{org4b1bbc5}
\begin{verbatim}
Running wavefunction...
Execution completed.
\end{verbatim}

\begin{center}
\includesvg[width=.9\linewidth]{./.ob-jupyter/4281df8825887f81fb3d0935751aec28797cb7ad}
\end{center}

\begin{minted}[breaklines=true,breakanywhere=true,frame=lines]{python}
papers(
    E         = [i for i in range(10)],
    m_val     = 1.0,
    V_0_val   = 0.0,
    h_bar_val = 1.0,
    a_val     = 1.5,
    b_val     = 1.5,
    T         = 10,
    sym       = 1,
    spin      = 0,
    grid_num  = 60
)
\end{minted}

\phantomsection
\label{org9a7e9ef}
\begin{verbatim}
Running wavefunction...
Execution completed.
\end{verbatim}

\begin{center}
\includesvg[width=.9\linewidth]{./.ob-jupyter/3cd678437884c2586eeca271ad874b12089a8ad1}
\end{center}

\begin{minted}[breaklines=true,breakanywhere=true,frame=lines]{python}
papers(
    E         = [i for i in range(10)],
    m_val     = 1.0,
    V_0_val   = 0.0,
    h_bar_val = 1.0,
    a_val     = 1.5,
    b_val     = 1.5,
    T         = 10,
    sym       =-1,
    spin      = 1,
    grid_num  = 60
)
\end{minted}

\phantomsection
\label{orgb256c18}
\begin{verbatim}
Running wavefunction...
Execution completed.
\end{verbatim}

\begin{center}
\includesvg[width=.9\linewidth]{./.ob-jupyter/72aad36672f4cf41145f22ec49034067cb591dbc}
\end{center}


\begin{minted}[breaklines=true,breakanywhere=true,frame=lines]{python}
papers(
    E         = [0,1],
    m_val     = 1.0,
    V_0_val   = 0.0,
    h_bar_val = 1.0,
    a_val     = 1.5,
    b_val     = 1.5,
    T         = 10,
    sym       =-1,
    spin      = 1,
    grid_num  = 60
)
\end{minted}

\phantomsection
\label{org9e641f2}
\begin{verbatim}
Running wavefunction...
Execution completed.
\end{verbatim}

\begin{center}
\includesvg[width=.9\linewidth]{./.ob-jupyter/b897205b67cbf7954bf311c3d7b20df98573325b}
\end{center}

\begin{minted}[breaklines=true,breakanywhere=true,frame=lines]{python}
papers(
    E         = [0,4],
    m_val     = 1.0,
    V_0_val   = 0.0,
    h_bar_val = 1.0,
    a_val     = 1.5,
    b_val     = 1.5,
    T         = 10,
    sym       =-1,
    spin      = 1,
    grid_num  = 60
)
\end{minted}

\phantomsection
\label{orgb8ce585}
\begin{verbatim}
Running wavefunction...
Execution completed.
\end{verbatim}

\begin{center}
\includesvg[width=.9\linewidth]{./.ob-jupyter/491d047a15f0fc00828f7c6948fe6f1fabef78a1}
\end{center}

\noindent\rule{\textwidth}{0.5pt}

A continuación, puesto que se han conseguido resultados que coinciden con la
literatura, consideraremos valores \(a \neq b\) para registrar lo que ocurre
cuando \(b >> a\), simulando las condiciones de un pozo finito.
\begin{minted}[breaklines=true,breakanywhere=true,frame=lines]{python}
papers(
    E = [0,2,21,22,23,24,25,40,41],
    m_val=1.0,
    V_0_val=10.0,
    h_bar_val=1.0,
    a_val=5,
    b_val=10,
    T=10,
    sym = -1,
    spin = 0,
    grid_num = 60
)
\end{minted}

\phantomsection
\label{orgec0779b}
\begin{verbatim}
Running wavefunction...
Execution completed.
\end{verbatim}

\begin{center}
\includesvg[width=.9\linewidth]{./.ob-jupyter/1462b572a8c1cad8a63b8df74c0785c07bd9652e}
\end{center}

\begin{minted}[breaklines=true,breakanywhere=true,frame=lines]{python}
papers(
    E         = [0,1,2,3,7,8,9],
    m_val     =1.0,
    V_0_val   =5.0,
    h_bar_val =1.0,
    a_val     =1.5,
    b_val     =10,
    T         =10,
    sym       = -1,
    spin      = 0
)
\end{minted}

\phantomsection
\label{orgf0eae15}
\begin{verbatim}
Running wavefunction...
Execution completed.
\end{verbatim}

\begin{center}
\includesvg[width=.9\linewidth]{./.ob-jupyter/a8870e994180a13bcd7d79f9525a37071fd7b8d2}
\end{center}

\begin{minted}[breaklines=true,breakanywhere=true,frame=lines]{python}
papers(
    E         = [2,3,7,8,9],
    m_val     =1.0,
    V_0_val   =5.0,
    h_bar_val =1.0,
    a_val     =1.5,
    b_val     =10,
    T         =10,
    sym       = -1,
    spin      = 1,
    grid_num  = 100
)
\end{minted}

\phantomsection
\label{org703d6b2}
\begin{verbatim}
Running wavefunction...
Execution completed.
\end{verbatim}

\begin{center}
\includesvg[width=.9\linewidth]{./.ob-jupyter/3ebbc9ee7f7c08cf2db885a3d67dc12ffbe8ad8f}
\end{center}

\begin{minted}[breaklines=true,breakanywhere=true,frame=lines]{python}
papers(
    E = [0,2,21,22,23,24,25,40,41],
    m_val=1.0,
    V_0_val=10.0,
    h_bar_val=1.0,
    a_val=5,
    b_val=10,
    T=10,
    sym = -1,
    spin = 1,
    grid_num = 60
)
\end{minted}

\phantomsection
\label{orgd1bf213}
\begin{verbatim}
Running wavefunction...
Execution completed.
\end{verbatim}

\begin{center}
\includesvg[width=.9\linewidth]{./.ob-jupyter/c907d15375a960cfa4bab0c01e85ab7679da32e4}
\end{center}

\begin{minted}[breaklines=true,breakanywhere=true,frame=lines]{python}
papers(
    E         = [5,10],
    m_val     = 1.0,
    V_0_val   = 0.0,
    h_bar_val = 1.0,
    a_val     = 3,
    b_val     = 6,
    T         = 1,
    sym       =-1,
    spin      = 0
)
\end{minted}

\phantomsection
\label{orgdde20d9}
\begin{verbatim}
Running wavefunction...
Execution completed.
\end{verbatim}

\begin{center}
\includesvg[width=.9\linewidth]{./.ob-jupyter/18c88c21d6faf1be4066285bd16b8157523277e2}
\end{center}

\begin{minted}[breaklines=true,breakanywhere=true,frame=lines]{python}
papers(
    E         = [15,23],
    m_val     = 1.0,
    V_0_val   = 10.0,
    h_bar_val = 1.0,
    a_val     = 5,
    b_val     = 10,
    T         = 1,
    sym       =-1,
    spin      = 0,
    grid_num  = 100
)
\end{minted}

\phantomsection
\label{orgf259aef}
\begin{verbatim}
Running wavefunction...
Execution completed.
\end{verbatim}

\begin{center}
\includesvg[width=.9\linewidth]{./.ob-jupyter/abacda005c9247071697f3521eb05a57f7ed4416}
\end{center}

\begin{minted}[breaklines=true,breakanywhere=true,frame=lines]{python}
papers(
    E         = [15,23],
    m_val     = 1.0,
    V_0_val   = 10.0,
    h_bar_val = 1.0,
    a_val     = 5,
    b_val     = 20,
    T         = 10,
    sym       =-1,
    spin      = 0,
    grid_num  = 100
)
\end{minted}

\phantomsection
\label{org565c9f7}
\begin{verbatim}
Running wavefunction...
Execution completed.
\end{verbatim}

\begin{center}
\includesvg[width=.9\linewidth]{./.ob-jupyter/c92d4e46362e50218e1963a842457265871f1ec9}
\end{center}

\noindent\rule{\textwidth}{0.5pt}
\section{\textbf{4. Conclusiones}}
\label{sec:org9c0ba91}
\begin{itemize}
\item Con nuestra metodología, fue posible reproducir satisfactoriamente los
resultados del paper [1].
\item En las gráficas de la densidad de probabilidad termalizada \(P_{th}\) unidimensionales, se
observa que a medida que se aumenta la temperatura del pozo finito, la
probabilidad de hallarse por fuera del mismo se homogeiniza con aquella de
encontrarse dentro. Este comportamiento se observa incluso cuando \(b >> a\),
y se pone de manifiesto en la forma en que se anula la capa fronteriza
(boundary layer) en estos casos.

\item En el caso de bosones y fermiones, también se observa la homogeinización de
la densidad de onda termalizada en las regiones adentro y afuera del pozo en
la medida en que se consideran más y más valores de energía, y temperaturas
más elevadas.

\item Al momento de considerarse varios niveles de energía, se hace mucho más
evidente el principio de exclusión de pauli en el caso de Fermiones, puesto
que las diagonales en las gráficas se anulan, contrario a lo que ocurre para
los bosones. Cuando se considera el espín, es posible encontrar las partículas
sobre las diagonales, pero la probabilidad sigue siendo reducida en
comparación con otras regiones.
\end{itemize}

\noindent\rule{\textwidth}{0.5pt}
\section{\textbf{5. Referencias}}
\label{sec:org0efe1bd}
{[}1] Miranda, E. N. (2019). Where are the particles when the box is hot?.
European Journal of Physics, 40(6), 065401.

{[}2] Hannabuss, K. (1997). An introduction to quantum theory (Vol. 1).
Clarendon Press.

{[}3] Bowers, P. L. (2020). Lectures on Quantum Mechanics: A Primer for
Mathematicians. Cambridge University Press.

{[}4] Scheck, F. (2016). Statistical Theory of Heat. Berlin: Springer.
\end{document}
